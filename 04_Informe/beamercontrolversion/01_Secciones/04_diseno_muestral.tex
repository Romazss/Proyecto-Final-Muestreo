% ============================================================================
% DISEÑO MUESTRAL
% ============================================================================
\section{Diseño Muestral}

\begin{frame}{Diseño Muestral}
\begin{block}{Características principales}
El diseño muestral de la Encuesta CASEN se caracteriza por ser:
\end{block}

\vspace{0.5cm}

\begin{enumerate}
    \item[\textcolor{celesteprincipal}{\textbf{1.}}] \textbf{Probabilístico:} Todas las unidades de la población tienen una probabilidad conocida y distinta de cero de ser seleccionadas
    
    \vspace{0.3cm}
    
    \item[\textcolor{celesteprincipal}{\textbf{2.}}] \textbf{Estratificado:} El territorio se divide en \textcolor{celesteoscuro}{\textbf{estratos geográficos y socioeconómicos}}
    
    \begin{itemize}
        \item Mejora la precisión de las estimaciones
        \item Asegura representatividad en distintos niveles (nacional, regional y urbano/rural)
    \end{itemize}
\end{enumerate}
\end{frame}

\begin{frame}{Diseño Muestral}
\begin{enumerate}
    \setcounter{enumi}{2}
    \item[\textcolor{celesteprincipal}{\textbf{3.}}] \textbf{Bietápico:} La selección se realiza en dos etapas:
    
    \vspace{0.4cm}
    
    \begin{block}{Primera etapa}
    Selección \textbf{sistemática con probabilidad proporcional al tamaño (PPT)} de las \textcolor{celesteoscuro}{\textbf{unidades primarias de muestreo (UPM)}} - conglomerados de viviendas
    \end{block}
    
    \vspace{0.3cm}
    
    \begin{block}{Segunda etapa}
    Selección \textbf{aleatoria simple (MAS)} de \textcolor{celesteoscuro}{\textbf{viviendas}} dentro de cada unidad primaria
    \end{block}
\end{enumerate}
\end{frame}

\begin{frame}{Tamaño Muestral}
\begin{block}{Consideraciones para el tamaño de muestra óptimo}
CASEN considera los siguientes criterios:
\end{block}

\vspace{0.3cm}

\begin{enumerate}
    \item[\textcolor{celesteprincipal}{\textbf{1.}}] Representatividad del \textbf{territorio nacional}, regiones y zonas urbanas/rurales
    
    \vspace{0.2cm}
    
    \item[\textcolor{celesteprincipal}{\textbf{2.}}] \textbf{Simulaciones} para optimizar precisión
\end{enumerate}     
\end{frame}

\begin{frame}{Tamaño de Muestra CASEN 2022}
\begin{table}[h]
\centering
\footnotesize
\renewcommand{\arraystretch}{1.3}
\begin{tabular}{>{\columncolor{celestefondo}}l l c c c c}
\toprule
\rowcolor{celesteprincipal}
\textcolor{white}{\textbf{}} & \textcolor{white}{\textbf{Nivel}}  & \textcolor{white}{\textbf{Tamaño}} & \textcolor{white}{\textbf{Error}} & \textcolor{white}{\textbf{Error}} & \textcolor{white}{\textbf{Tamaño con}} \\
\rowcolor{celesteprincipal}
\textcolor{white}{\textbf{}} & \textcolor{white}{\textbf{}}  & \textcolor{white}{\textbf{Objetivo}} & \textcolor{white}{\textbf{Absoluto}} & \textcolor{white}{\textbf{Relativo}} & \textcolor{white}{\textbf{Sobremuestreo}} \\
\midrule
\rowcolor{white}
 & \textbf{País}  & 71.028 & 0,4\% & 3,3\% & 106.856 \\
\rowcolor{celestefondo}
 & Urbano & 56.905 & 0,5\% & 4,6\% & 87.252  \\
\rowcolor{white}
 & Rural  & 14.123 & 1,3\% & 9,2\% & 19.604  \\
\bottomrule
\end{tabular}
\end{table}

\vspace{0.2cm}
\small
\textcolor{grisoscuro}{\textit{Fuente: Manual Metodológico CASEN 2022, p. 35}}
\end{frame}

\begin{frame}{Marco Muestral}
\begin{block}{Construcción del Marco}
A partir del \textcolor{celesteoscuro}{\textbf{MMV 2020}} se elabora el marco de selección para CASEN 2022
\end{block}

\vspace{0.3cm}

\begin{itemize}
    \item Conformado por \textbf{335 comunas} definidas para el nuevo diseño de la Encuesta Nacional de Empleo (ENE 2020)
    
    \vspace{0.2cm}
    
    \item Las UPM del marco muestral están estratificadas por:
    \begin{enumerate}
        \item[\textcolor{celesteprincipal}{$\bullet$}] \textbf{Geografía} (comunas)
        \item[\textcolor{celesteprincipal}{$\bullet$}] \textbf{Áreas} (urbana–rural)
        \item[\textcolor{celesteprincipal}{$\bullet$}] \textbf{Nivel socioeconómico (NSE)} - construida a partir de la clasificación del MMV 2020
    \end{enumerate}
\end{itemize}
\end{frame}

\begin{frame}{Nivel de Inferencia}
\begin{block}{Objetivo Analítico}
Producir inferencias hacia la población que reside en \textcolor{celesteoscuro}{\textbf{viviendas particulares ocupadas}} (elegibles)
\end{block}

\vspace{0.4cm}

\begin{alertblock}{Exclusiones}
No se consideran para fines analíticos las viviendas \textbf{no elegibles}:
\end{alertblock}

\vspace{0.2cm}

\begin{columns}[t]
\begin{column}{0.48\textwidth}
\begin{itemize}
    \item Oficinas de empresas
    \item Viviendas abandonadas
\end{itemize}
\end{column}
\begin{column}{0.48\textwidth}
\begin{itemize}
    \item Viviendas de veraneo
    \item Viviendas demolidas
\end{itemize}
\end{column}
\end{columns}
\end{frame}
