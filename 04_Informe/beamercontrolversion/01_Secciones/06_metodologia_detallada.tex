% ============================================================================
% METODOLOGÍA DETALLADA
% ============================================================================
\section{Metodología Detallada}

\begin{frame}{Estimador de Horvitz-Thompson}
\begin{block}{Estimador de la media ponderada}
Para estimar medias y proporciones bajo un diseño muestral complejo, utilizamos el estimador de Horvitz-Thompson ajustado por el factor de expansión:
\end{block}

\vspace{0.3cm}

\begin{equation*}
\textcolor{celesteoscuro}{\hat{\bar{Y}} = \frac{\sum_{i \in s} \frac{y_i}{\pi_i}}{\sum_{i \in s} \frac{1}{\pi_i}} \approx
\frac{\sum_{i \in s} expr_i \, y_i}{\sum_{i \in s} expr_i}}
\end{equation*}

\vspace{0.3cm}

donde:
\begin{itemize}
    \item $y_i$ corresponde a la \textbf{variable de interés}
    \item $expr_i$ es el \textcolor{celesteoscuro}{\textbf{factor de expansión}} provisto por CASEN 2022
    \item $\pi_i$ es la probabilidad de inclusión de la unidad $i$
\end{itemize}
\end{frame}

\begin{frame}{Varianza del Estimador}
\begin{block}{Varianza e intervalos de confianza}
La varianza del estimador se calcula considerando el \textbf{efecto del diseño} (estratificación y conglomeración):
\end{block}

\vspace{0.3cm}

\begin{equation*}
\textcolor{celesteoscuro}{V(\hat{Y}) = \sum_h \left(1 - \frac{n_h}{N_h}\right) \frac{s_h^2}{n_h}}
\end{equation*}

\vspace{0.4cm}

\begin{alertblock}{Ponderadores}
Se usará el factor de expansión \texttt{expr} para corregir las \textbf{probabilidades desiguales de selección} en el diseño muestral
\end{alertblock}
\end{frame}

\begin{frame}{Software y Herramientas}
\begin{block}{Implementación en R}
Los métodos se implementarán utilizando los siguientes paquetes:
\end{block}

\vspace{0.4cm}

\begin{columns}[t]
\begin{column}{0.45\textwidth}
\begin{itemize}
    \item[\textcolor{celesteprincipal}{$\checkmark$}] \texttt{survey} y \texttt{srvyr}
    \begin{itemize}
        \item Estimaciones bajo diseño complejo
    \end{itemize}
\end{itemize}
\end{column}

\begin{column}{0.45\textwidth}
\begin{itemize}
    \item[\textcolor{celesteprincipal}{$\checkmark$}] \texttt{ggplot2}
    \begin{itemize}
        \item Visualización de datos
    \end{itemize}
    \item[\textcolor{celesteprincipal}{$\checkmark$}] \texttt{dplyr}
    \begin{itemize}
        \item Procesamiento de datos
    \end{itemize}
\end{itemize}
\end{column}
\end{columns}
\end{frame}
