% ============================================================================
% METODOLOGÍA DETALLADA
% ============================================================================
\section{Metodología Detallada}

\begin{frame}{Estimadores}
\begin{block}{Estimador de Horvitz-Thompson (Media ponderada)}
$$\textcolor{celesteoscuro}{\hat{\bar{Y}} = \frac{\sum_{i \in s} w_i \, y_i}{\sum_{i \in s} w_i}}$$
donde $w_i = \texttt{expr}_i$ es el factor de expansión de CASEN 2022
\end{block}

\vspace{0.4cm}

\begin{block}{Aplicación en CASEN}
\begin{itemize}
    \item Corrige probabilidades desiguales de selección (PPT en 1ª etapa)
    \item Factor $w_i$ incluye ajustes de no-respuesta y calibración
    \item Se usa para medias, totales y proporciones poblacionales
\end{itemize}
\end{block}

\vspace{0.3cm}

\footnotesize
\textcolor{grisoscuro}{\textit{Ref: Diseño Muestral CASEN 2022, ecs. (43)–(44)}}
\end{frame}

\begin{frame}{Varianzas y Errores Estándar}
\begin{block}{Varianzas: EVCU/WR oficial (sin FPC, $f_h=0$)}
$$\textcolor{celesteoscuro}{V(\hat{\bar{Y}}) = \sum_{h} \frac{1}{n_h(n_h-1)} \sum_{i=1}^{n_h} \left(z_{hi} - \bar{z}_h\right)^2}$$
\footnotesize
donde $z_{hi} = w_{hi}(y_{hi} - \hat{\bar{Y}})$ es el residuo ponderado en estrato $h$
\end{block}

\vspace{0.2cm}

\begin{block}{Implementación}
\footnotesize
Variables de diseño: \texttt{varstrat} (estratos) y \texttt{varunit} (UPM). Linealización de Taylor para totales/razones.
\end{block}

\vspace{0.2cm}

\begin{alertblock}{Nota: Aproximación WR}
\scriptsize
Aunque la 1ª etapa es \textbf{sin reemplazo} (PPT sistemática), se usa fórmula \textbf{WR sin FPC} como aproximación conservadora estándar en encuestas complejas (fracciones de muestreo pequeñas $\Rightarrow$ efecto despreciable).
\end{alertblock}

\vspace{0.1cm}

\footnotesize
\textcolor{grisoscuro}{\textit{Ref: Diseño Muestral CASEN 2022, ecs. (45)–(47)}}
\end{frame}

\begin{frame}{Software y Ponderadores}
\begin{columns}[t]
\begin{column}{0.48\textwidth}
\textcolor{celesteoscuro}{\textbf{Software Principal (R):}}
\begin{itemize}
    \footnotesize
    \item \texttt{survey} / \texttt{srvyr}: Diseño complejo, svyglm()
    \item \texttt{mice} / \texttt{mitools}: Imputación múltiple
    \item \texttt{sandwich}: Errores robustos
\end{itemize}

\vspace{0.2cm}

\textcolor{celesteoscuro}{\textbf{Soporte (Python):}}
\begin{itemize}
    \footnotesize
    \item \texttt{pandas}, \texttt{numpy}: Procesamiento
    \item Solo para QA descriptivo
\end{itemize}
\end{column}

\begin{column}{0.48\textwidth}
\textcolor{celesteoscuro}{\textbf{Ponderadores:}}
\begin{itemize}
    \footnotesize
    \item Variable: \texttt{expr} (CASEN 2022)
    \item Corrige probabilidades desiguales de selección
    \item Aplicado en todos los estimadores
\end{itemize}

\vspace{0.2cm}

\begin{alertblock}{Inferencia}
\footnotesize
Toda inferencia por diseño se realiza en R con \texttt{survey}
\end{alertblock}
\end{column}
\end{columns}
\end{frame}

\begin{frame}{Regresiones con Diseño Complejo}
\begin{block}{Método: \texttt{survey::svyglm()} con Linealización de Taylor}
Declarar diseño (\texttt{svydesign}) antes de estimar. Respeta estratos, UPM y pesos.
\end{block}

\vspace{0.2cm}

\begin{columns}[t]
\begin{column}{0.48\textwidth}
\textcolor{celesteoscuro}{\textbf{Modelos GLM:}}
\begin{itemize}
    \footnotesize
    \item Lineal: \texttt{svyglm(y \textasciitilde\  x)}
    \item Logística: \texttt{family=quasibinomial()}
    \item Poisson: \texttt{family=quasipoisson()}
\end{itemize}

\vspace{0.2cm}

\textcolor{celesteoscuro}{\textbf{Errores Estándar:}}
\begin{itemize}
    \footnotesize
    \item \textbf{Estándar:} EE por diseño (Taylor)
    \item \textbf{Sensibilidad:} HC/cluster documentada
\end{itemize}
\end{column}

\begin{column}{0.48\textwidth}
% Sección 'Grados de Libertad' eliminada según solicitud del autor.
% (Originalmente: DF = degf(design) = #PSU - #estratos; se quitó para simplificar la diapositiva.)
\vspace{0.1cm}

\textcolor{celesteoscuro}{\textbf{Inferencia:}}
\begin{itemize}
    \footnotesize
    \item $\alpha = 0.05$ | IC 95\% por diseño
\end{itemize}
\end{column}
\end{columns}
\end{frame}

\begin{frame}{Manejo de Datos Faltantes}
\begin{block}{Política Oficial CASEN}
\textbf{CASEN no imputa} excepto variables de ingreso (metodología Mideplan oficial).
\end{block}

\vspace{0.2cm}

\begin{columns}[t]
\begin{column}{0.48\textwidth}
\textcolor{celesteoscuro}{\textbf{Análisis Principal:}}
\begin{itemize}
    \footnotesize
    \item \texttt{subset()} sobre diseño (preserva pesos)
    \item Reportar \% missingness por variable
    \item Justificar exclusión si >5\%
\end{itemize}

\vspace{0.2cm}

\textcolor{celesteoscuro}{\textbf{Sensibilidad (si aplica):}}
\begin{itemize}
    \footnotesize
    \item IM solo si \% missing sustantivo
    \item Justificar supuesto MAR explícitamente
\end{itemize}
\end{column}

\begin{column}{0.48\textwidth}
\textcolor{celesteoscuro}{\textbf{Flujo IM + Diseño:}}
\begin{enumerate}
    \footnotesize
    \item \texttt{mice::mice(data, m=20)}
    \item \texttt{svydesign} por imputación
    \item Ajustar modelo en cada una
    \item \texttt{MIcombine()} (Reglas de Rubin)
\end{enumerate}

\vspace{0.1cm}

\begin{alertblock}{\footnotesize Nota}
\footnotesize
Documentar mecanismo de missing y comparar con casos completos
\end{alertblock}
\end{column}
\end{columns}
\end{frame}
