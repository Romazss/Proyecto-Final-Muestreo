% ============================================================================
% PLAN DE ANÁLISIS
% ============================================================================
\section{Plan de Análisis}

\begin{frame}{Objetivo 1: Brecha Salarial de Género}
\begin{block}{Objetivo}
Cuantificar y explicar las diferencias salariales por género controlando por factores socioeconómicos, laborales y educativos
\end{block}

\vspace{0.2cm}

\begin{alertblock}{Hipótesis de Investigación}
\footnotesize
\textbf{H1:} Existe una brecha salarial significativa entre hombres y mujeres ($\mu_{\text{hombre}} > \mu_{\text{mujer}}$), \\
\phantom{\textbf{H1:}} incluso controlando por educación, edad, ocupación y composición del hogar
\end{alertblock}

\vspace{0.2cm}

\begin{columns}[t]
\begin{column}{0.48\textwidth}
\textcolor{celesteoscuro}{\textbf{Variables principales:}}
\begin{itemize}
    \footnotesize
    \item \texttt{sexo}: Sexo de la persona
    \item \texttt{ytrabajocorh}: Ingreso del trabajo principal corregido
    \item \texttt{esc}: Años de educación formal
\end{itemize}
\end{column}

\begin{column}{0.48\textwidth}
\textcolor{celesteoscuro}{\textbf{Variables de control:}}
\begin{itemize}
    \footnotesize
    \item \texttt{edad}: Edad de la persona
    \item \texttt{oficio4\_08}: Ocupación
    \item \texttt{tot\_per\_h}: Total personas en el hogar
\end{itemize}
\end{column}
\end{columns}
\end{frame}

\begin{frame}{Metodología - Objetivo 1}
\begin{block}{Estrategia de Contrastación}
Estimación en tres etapas con creciente control de confusores
\end{block}

\vspace{0.2cm}

\begin{enumerate}
    \item[\textcolor{celesteprincipal}{\textbf{1.}}] \textbf{Análisis bivariado:} Test de diferencia de medias
    \begin{itemize}
        \footnotesize
        \item \texttt{svyttest(ytrabajocorh \textasciitilde\  sexo)} 
        \item Reporta brecha bruta sin controles
    \end{itemize}
    
    \vspace{0.2cm}
    
    \item[\textcolor{celesteprincipal}{\textbf{2.}}] \textbf{Modelo ajustado (controles socioeconómicos):}
    \begin{itemize}
        \footnotesize
        \item \texttt{svyglm(ytrabajocorh \textasciitilde\  sexo + esc + edad)}
        \item Aísla efecto directo del género
    \end{itemize}
    
    \vspace{0.2cm}
    
    \item[\textcolor{celesteprincipal}{\textbf{3.}}] \textbf{Modelo completo (controles laborales):}
    \begin{itemize}
        \footnotesize
        \item Incluye \texttt{oficio4\_08} y \texttt{tot\_per\_h}
        \item Evalúa mediación por segregación ocupacional
    \end{itemize}
\end{enumerate}

\vspace{0.1cm}
\footnotesize
\textcolor{grisoscuro}{\textit{Decisión: Rechazar H0 si p < 0.05 en modelo completo con coeficiente $\beta_{\text{sexo}} < 0$}}
\end{frame}

\begin{frame}{Objetivo 2: Distribución de la Pobreza}
\begin{block}{Objetivo}
Cuantificar disparidades territorial-educativas en la prevalencia de pobreza y sus determinantes estructurales
\end{block}

\vspace{0.2cm}

\begin{alertblock}{Hipótesis de Investigación}
\footnotesize
\textbf{H2a:} La tasa de pobreza en zona rural es significativamente mayor que en zona urbana ($p_{\text{rural}} > p_{\text{urbano}}$) \\[0.1cm]
\textbf{H2b:} La educación reduce la probabilidad de pobreza, con efecto más pronunciado en zonas urbanas
\end{alertblock}

\vspace{0.2cm}

\begin{columns}[t]
\begin{column}{0.48\textwidth}
\textcolor{celesteoscuro}{\textbf{Variables principales:}}
\begin{itemize}
    \footnotesize
    \item \texttt{pobreza}: Condición de pobreza
    \item \texttt{ytotcorh}: Ingreso total corregido
    \item \texttt{zona}: Rural/Urbana
\end{itemize}
\end{column}

\begin{column}{0.48\textwidth}
\textcolor{celesteoscuro}{\textbf{Variables de control:}}
\begin{itemize}
    \footnotesize
    \item \texttt{esc}: Años de educación
    \item \texttt{región}: Región del país
    \item \texttt{edad}: Edad de la persona
    \item \texttt{expr}: Factor de expansión
\end{itemize}
\end{column}
\end{columns}
\end{frame}

\begin{frame}{Metodología - Objetivo 2}
\begin{block}{Estrategia de Contrastación Secuencial}
Tres análisis complementarios para triangular evidencia
\end{block}

\vspace{0.2cm}

\begin{enumerate}
    \item[\textcolor{celesteprincipal}{\textbf{1.}}] \textbf{H2a - Test bivariado:} 
    \begin{itemize}
        \footnotesize
        \item \texttt{svychisq(\textasciitilde\  pobreza + zona)} - Prueba de independencia
        \item \texttt{svyttest(ytotcorh \textasciitilde\  zona)} - Diferencia de ingresos
    \end{itemize}
    
    \vspace{0.1cm}
    
    \item[\textcolor{celesteprincipal}{\textbf{2.}}] \textbf{H2b - Modelo logístico principal:}
    \begin{itemize}
        \footnotesize
        \item \texttt{svyglm(pobreza \textasciitilde\  zona + esc + edad, family=quasibinomial())}
        \item Comparar OR de educación entre zonas
    \end{itemize}
    
    \vspace{0.1cm}
    
    \item[\textcolor{celesteprincipal}{\textbf{3.}}] \textbf{H2b - Análisis de mediación:}
    \begin{itemize}
        \footnotesize
        \item Estimar efecto indirecto de zona vía educación
        \item Test de Sobel para significancia del efecto mediador
    \end{itemize}
\end{enumerate}
\end{frame}
