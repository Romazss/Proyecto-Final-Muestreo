% ============================================================================
% ANÁLISIS DE GAPS - VERIFICACIÓN DE CONTENIDOS MÍNIMOS
% Documento para identificar información faltante en la presentación
% ============================================================================

\documentclass[11pt,aspectratio=169]{beamer}
\usetheme{Madrid}
\usefonttheme{serif}

% Cargar preámbulo
% ============================================================================
% 00_preambulo_beamer.tex
% ============================================================================
% Preámbulo común para la presentación Beamer
% Contiene paquetes, configuración de estilos y colores

% ============================================================================
% PAQUETES
% ============================================================================
\usepackage[utf8]{inputenc}
\usepackage[spanish]{babel}
\usepackage[T1]{fontenc}
\usepackage{multirow}
\usepackage{booktabs}
\usepackage{amsmath}
\usepackage{amsfonts}
\usepackage{amssymb}
\usepackage{graphicx}
\usepackage{tikz}
\usepackage{xcolor}
\usepackage{colortbl}

% ============================================================================
% DEFINICIÓN DE COLORES PUC (Coherente con el informe)
% ============================================================================
\definecolor{celesteprincipal}{RGB}{0,150,200}
\definecolor{celestesuave}{RGB}{135,206,235}
\definecolor{celesteclaro}{RGB}{173,216,230}
\definecolor{celesteoscuro}{RGB}{0,105,148}
\definecolor{celestefondo}{RGB}{230,245,255}
\definecolor{grisoscuro}{RGB}{64,64,64}

% ============================================================================
% CONFIGURACIÓN DE COLORES DEL TEMA
% ============================================================================
\setbeamercolor{structure}{fg=celesteoscuro}
\setbeamercolor{palette primary}{bg=celesteprincipal,fg=white}
\setbeamercolor{palette secondary}{bg=celestesuave,fg=white}
\setbeamercolor{palette tertiary}{bg=celesteoscuro,fg=white}
\setbeamercolor{palette quaternary}{bg=celesteclaro,fg=grisoscuro}

\setbeamercolor{titlelike}{parent=palette primary}
\setbeamercolor{frametitle}{bg=celesteprincipal,fg=white}
\setbeamercolor{block title}{bg=celesteprincipal,fg=white}
\setbeamercolor{block body}{bg=celestefondo,fg=grisoscuro}
\setbeamercolor{item}{fg=celesteoscuro}
\setbeamercolor{section in toc}{fg=celesteoscuro}

% ============================================================================
% CONFIGURACIÓN DE PLANTILLAS
% ============================================================================
\setbeamertemplate{caption}[numbered]
\setbeamertemplate{navigation symbols}{}
\setbeamertemplate{footline}[frame number]
\setbeamertemplate{itemize items}[circle]
\setbeamertemplate{enumerate items}[default]

% Configuración de bloques con bordes redondeados
\setbeamertemplate{blocks}[rounded][shadow=true]

% ============================================================================
% INFORMACIÓN DEL DOCUMENTO
% ============================================================================
\title[Métodos - CASEN 2022]{Análisis de Desigualdades Socioeconómicas}
\subtitle{Encuesta CASEN 2022 - Métodos y Diseño Muestral}
\author[Grupo 4]{
    Alexander Pinto \and 
    Esteban Román \and 
    Julián Vargas \and 
    Francisca Sepúlveda
}
\institute[PUC]{
    \Large Pontificia Universidad Católica de Chile \\[0.3cm]
    \small EYP2417 - Muestreo \\
    \small Grupo 4
}
\date{11 de noviembre de 2025}

% Logo en el pie de página
\logo{\includegraphics[height=0.8cm]{../03_Logos/logo_kovan.jpg}}

% ============================================================================
% CONFIGURACIÓN DE BIBLIOGRAFÍA
% ============================================================================
\bibliographystyle{apalike}


% Definir colores adicionales para gaps
\definecolor{gapurgente}{RGB}{220,20,60}      % Rojo para gaps críticos
\definecolor{gapmedio}{RGB}{255,165,0}        % Naranja para gaps medios
\definecolor{gapbajo}{RGB}{255,215,0}         % Amarillo para gaps menores
\definecolor{completo}{RGB}{34,139,34}        % Verde para secciones completas

\title{Análisis de Gaps}
\subtitle{Verificación de Contenidos Mínimos}
\author{Grupo 4}
\date{Noviembre 2025}

\begin{document}

% ============================================================================
% PORTADA
% ============================================================================
\begin{frame}[plain]
\titlepage
\end{frame}

% ============================================================================
% TABLA DE CONTENIDOS
% ============================================================================
\begin{frame}{Índice de Verificación}
\tableofcontents
\end{frame}

% ============================================================================
% SECCIÓN 1: DISEÑO MUESTRAL
% ============================================================================
\section{Diseño Muestral}

\begin{frame}{1. Diseño Muestral - Estado General}
\begin{block}{Resumen de Completitud}
Estado de verificación de los 4 componentes del diseño muestral
\end{block}

\vspace{0.3cm}

\begin{tabular}{l c l}
\toprule
\textbf{Componente} & \textbf{Estado} & \textbf{Completitud} \\
\midrule
Tipo de Diseño & \textcolor{completo}{✓} & 95\% \\
Tamaño de Muestra & \textcolor{gapmedio}{⚠} & 70\% \\
Marco Muestral & \textcolor{completo}{✓} & 90\% \\
Nivel de Inferencia & \textcolor{gapmedio}{⚠} & 65\% \\
\bottomrule
\end{tabular}

\vspace{0.3cm}

\begin{alertblock}{Completitud General}
\textbf{80\%} - Requiere ajustes en 2 componentes
\end{alertblock}
\end{frame}

\begin{frame}{1.1 Tipo de Diseño - Análisis Detallado}
\begin{block}{\textcolor{completo}{✓} Información Presente (95\%)}
\end{block}

\begin{itemize}
    \item[\textcolor{completo}{✓}] Diseño probabilístico claramente explicado
    \item[\textcolor{completo}{✓}] Estratificación geográfica y socioeconómica detallada
    \item[\textcolor{completo}{✓}] Diseño bietápico con etapas bien definidas
    \item[\textcolor{completo}{✓}] PPT en primera etapa y MAS en segunda etapa
\end{itemize}

\vspace{0.3cm}

\begin{alertblock}{\textcolor{gapmedio}{⚠} Gap Menor Identificado (5\%)}
\end{alertblock}

\begin{itemize}
    \item[\textcolor{gapmedio}{⚠}] \textbf{Falta:} Justificación explícita de por qué se eligió diseño bietápico vs. otras alternativas
    \item[\textcolor{gapmedio}{⚠}] \textbf{Falta:} Mención del efecto de diseño (DEFF) estimado
\end{itemize}
\end{frame}

\begin{frame}{1.2 Tamaño de Muestra - Análisis Detallado}
\begin{block}{\textcolor{completo}{✓} Información Presente (70\%)}
\end{block}

\begin{itemize}
    \item[\textcolor{completo}{✓}] Tabla con tamaño de muestra por nivel (nacional, urbano, rural)
    \item[\textcolor{completo}{✓}] Errores absolutos y relativos presentados
    \item[\textcolor{completo}{✓}] Mención de sobremuestreo
    \item[\textcolor{completo}{✓}] Criterios generales para tamaño de muestra
\end{itemize}

\vspace{0.3cm}

\begin{alertblock}{\textcolor{gapmedio}{⚠} Gaps Identificados (30\%)}
\end{alertblock}

\begin{itemize}
    \item[\textcolor{gapmedio}{⚠}] \textbf{Falta:} Explicación de cómo se calcularon los errores (fórmulas)
    \item[\textcolor{gapmedio}{⚠}] \textbf{Falta:} Justificación del nivel de confianza usado (95\%?)
    \item[\textcolor{gapurgente}{✗}] \textbf{Falta:} Tasa de respuesta esperada vs. obtenida
    \item[\textcolor{gapurgente}{✗}] \textbf{Falta:} Impacto de no respuesta en precisión
\end{itemize}
\end{frame}

\begin{frame}{1.3 Marco Muestral - Análisis Detallado}
\begin{block}{\textcolor{completo}{✓} Información Presente (90\%)}
\end{block}

\begin{itemize}
    \item[\textcolor{completo}{✓}] Mención del MMV 2020 como base
    \item[\textcolor{completo}{✓}] 335 comunas identificadas
    \item[\textcolor{completo}{✓}] Estratificación por geografía, área y NSE
    \item[\textcolor{completo}{✓}] Conexión con ENE 2020
\end{itemize}

\vspace{0.3cm}

\begin{alertblock}{\textcolor{gapmedio}{⚠} Gaps Menores (10\%)}
\end{alertblock}

\begin{itemize}
    \item[\textcolor{gapmedio}{⚠}] \textbf{Falta:} Fecha de actualización del marco (¿2020 está actualizado para 2022?)
    \item[\textcolor{gapmedio}{⚠}] \textbf{Falta:} Número total de UPM en el marco
    \item[\textcolor{gapmedio}{⚠}] \textbf{Falta:} Cobertura del marco (¿qué \% de la población?)
\end{itemize}
\end{frame}

\begin{frame}{1.4 Nivel de Inferencia - Análisis Detallado}
\begin{block}{\textcolor{completo}{✓} Información Presente (65\%)}
\end{block}

\begin{itemize}
    \item[\textcolor{completo}{✓}] Objetivo: viviendas particulares ocupadas
    \item[\textcolor{completo}{✓}] Exclusión de viviendas no elegibles claramente listada
    \item[\textcolor{completo}{✓}] Mención de áreas especiales excluidas (introducción)
\end{itemize}

\vspace{0.3cm}

\begin{alertblock}{\textcolor{gapurgente}{✗} Gaps Importantes (35\%)}
\end{alertblock}

\begin{itemize}
    \item[\textcolor{gapurgente}{✗}] \textbf{Falta:} Niveles específicos de inferencia (¿nacional? ¿regional? ¿comunal?)
    \item[\textcolor{gapurgente}{✗}] \textbf{Falta:} ¿Se puede inferir a nivel de zona urbana/rural por región?
    \item[\textcolor{gapmedio}{⚠}] \textbf{Falta:} Limitaciones de inferencia explícitas
    \item[\textcolor{gapmedio}{⚠}] \textbf{Falta:} ¿Qué \% de la población queda excluida?
\end{itemize}
\end{frame}

% ============================================================================
% SECCIÓN 2: PLAN DE ANÁLISIS
% ============================================================================
\section{Plan de Análisis}

\begin{frame}{2. Plan de Análisis - Estado General}
\begin{block}{Resumen de Completitud}
Estado de verificación de los 3 componentes del plan de análisis
\end{block}

\vspace{0.3cm}

\begin{tabular}{l c l}
\toprule
\textbf{Componente} & \textbf{Estado} & \textbf{Completitud} \\
\midrule
Objetivos Específicos & \textcolor{completo}{✓} & 85\% \\
Variables Involucradas & \textcolor{completo}{✓} & 90\% \\
Métodos Propuestos & \textcolor{gapmedio}{⚠} & 70\% \\
\bottomrule
\end{tabular}

\vspace{0.3cm}

\begin{alertblock}{Completitud General}
\textbf{82\%} - Requiere ajustes en métodos propuestos
\end{alertblock}
\end{frame}

\begin{frame}{2.1 Objetivos Específicos - Análisis Detallado}
\begin{block}{\textcolor{completo}{✓} Información Presente (85\%)}
\end{block}

\begin{itemize}
    \item[\textcolor{completo}{✓}] Objetivo 1: Brecha salarial de género claramente definido
    \item[\textcolor{completo}{✓}] Objetivo 2: Distribución de pobreza bien articulado
    \item[\textcolor{completo}{✓}] Ambos objetivos son medibles y específicos
\end{itemize}

\vspace{0.3cm}

\begin{alertblock}{\textcolor{gapmedio}{⚠} Gaps Identificados (15\%)}
\end{alertblock}

\begin{itemize}
    \item[\textcolor{gapmedio}{⚠}] \textbf{Falta:} Conexión explícita entre objetivos y diseño muestral
    \item[\textcolor{gapmedio}{⚠}] \textbf{Falta:} ¿Por qué estos objetivos son importantes para CASEN?
    \item[\textcolor{gapmedio}{⚠}] \textbf{Falta:} Hipótesis de investigación explícitas
\end{itemize}
\end{frame}

\begin{frame}{2.2 Variables Involucradas - Análisis Detallado}
\begin{block}{\textcolor{completo}{✓} Información Presente (90\%)}
\end{block}

\textbf{Objetivo 1 - Brecha Salarial:}
\begin{itemize}
    \item[\textcolor{completo}{✓}] Variable dependiente: \texttt{ytrabajocorh}
    \item[\textcolor{completo}{✓}] Variables principales: sexo, esc, edad, oficio4\_08
    \item[\textcolor{completo}{✓}] Variable estructural: tot\_per\_h
\end{itemize}

\textbf{Objetivo 2 - Pobreza:}
\begin{itemize}
    \item[\textcolor{completo}{✓}] Variable dependiente: \texttt{pobreza}
    \item[\textcolor{completo}{✓}] Variables principales: zona, ytotcorh, esc, región, edad
    \item[\textcolor{completo}{✓}] Ponderador: expr
\end{itemize}

\vspace{0.2cm}

\begin{alertblock}{\textcolor{gapmedio}{⚠} Gap Menor (10\%)}
\end{alertblock}

\begin{itemize}
    \item[\textcolor{gapmedio}{⚠}] \textbf{Falta:} Descripción de escalas de medición (continua, categórica, ordinal)
\end{itemize}
\end{frame}

\begin{frame}{2.3 Métodos Propuestos - Análisis Detallado (1/2)}
\begin{block}{\textcolor{completo}{✓} Información Presente (70\%)}
\end{block}

\textbf{Objetivo 1:}
\begin{itemize}
    \item[\textcolor{completo}{✓}] Análisis descriptivo mencionado
    \item[\textcolor{completo}{✓}] Regresión lineal ponderada propuesta
    \item[\textcolor{completo}{✓}] Uso de factores de expansión mencionado
\end{itemize}

\textbf{Objetivo 2:}
\begin{itemize}
    \item[\textcolor{completo}{✓}] Cálculo de porcentajes de pobreza
    \item[\textcolor{completo}{✓}] Comparación de ingresos promedio
    \item[\textcolor{completo}{✓}] Pruebas de hipótesis ponderadas
    \item[\textcolor{completo}{✓}] Regresión logística propuesta
\end{itemize}
\end{frame}

\begin{frame}{2.3 Métodos Propuestos - Análisis Detallado (2/2)}
\begin{alertblock}{\textcolor{gapurgente}{✗} Gaps Importantes (30\%)}
\end{alertblock}

\begin{itemize}
    \item[\textcolor{gapurgente}{✗}] \textbf{Falta:} ¿Cómo se incorpora el diseño complejo en las regresiones?
    \item[\textcolor{gapurgente}{✗}] \textbf{Falta:} ¿Qué paquetes de R/Python se usarán para cada método?
    \item[\textcolor{gapurgente}{✗}] \textbf{Falta:} ¿Cómo se calculan errores estándar robustos?
    \item[\textcolor{gapmedio}{⚠}] \textbf{Falta:} Nivel de significancia para pruebas de hipótesis
    \item[\textcolor{gapmedio}{⚠}] \textbf{Falta:} ¿Cómo se manejan valores missing?
    \item[\textcolor{gapmedio}{⚠}] \textbf{Falta:} ¿Se harán ajustes por comparaciones múltiples?
\end{itemize}
\end{frame}

% ============================================================================
% SECCIÓN 3: METODOLOGÍA DETALLADA
% ============================================================================
\section{Metodología Detallada}

\begin{frame}{3. Metodología Detallada - Estado General}
\begin{block}{Resumen de Completitud}
Estado de verificación de los 5 componentes de metodología detallada
\end{block}

\vspace{0.3cm}

\begin{tabular}{l c l}
\toprule
\textbf{Componente} & \textbf{Estado} & \textbf{Completitud} \\
\midrule
Estimadores Usados & \textcolor{completo}{✓} & 85\% \\
Varianzas & \textcolor{gapmedio}{⚠} & 75\% \\
Ponderadores & \textcolor{completo}{✓} & 90\% \\
Software & \textcolor{completo}{✓} & 95\% \\
Limitaciones & \textcolor{gapurgente}{✗} & 0\% \\
\bottomrule
\end{tabular}

\vspace{0.3cm}

\begin{alertblock}{Completitud General}
\textbf{69\%} - \textcolor{gapurgente}{¡GAP CRÍTICO en limitaciones!}
\end{alertblock}
\end{frame}

\begin{frame}{3.1 Estimadores Usados - Análisis Detallado}
\begin{block}{\textcolor{completo}{✓} Información Presente (85\%)}
\end{block}

\begin{itemize}
    \item[\textcolor{completo}{✓}] Estimador de Horvitz-Thompson presentado
    \item[\textcolor{completo}{✓}] Fórmula matemática correcta
    \item[\textcolor{completo}{✓}] Notación bien definida ($y_i$, $\pi_i$, $expr_i$)
    \item[\textcolor{completo}{✓}] Aproximación con factor de expansión mostrada
    \item[\textcolor{completo}{✓}] Explicación de componentes incluida
\end{itemize}

\vspace{0.3cm}

\begin{alertblock}{\textcolor{gapmedio}{⚠} Gaps Menores (15\%)}
\end{alertblock}

\begin{itemize}
    \item[\textcolor{gapmedio}{⚠}] \textbf{Falta:} Relación entre $\pi_i$ y $expr_i$ (¿son inversos?)
    \item[\textcolor{gapmedio}{⚠}] \textbf{Falta:} ¿Cómo se obtienen las probabilidades de inclusión?
    \item[\textcolor{gapmedio}{⚠}] \textbf{Falta:} Estimador para totales además de medias
\end{itemize}
\end{frame}

\begin{frame}{3.2 Varianzas - Análisis Detallado}
\begin{block}{\textcolor{completo}{✓} Información Presente (75\%)}
\end{block}

\begin{itemize}
    \item[\textcolor{completo}{✓}] Fórmula de varianza presentada
    \item[\textcolor{completo}{✓}] Consideración del efecto del diseño mencionada
    \item[\textcolor{completo}{✓}] Estratificación incluida en fórmula
    \item[\textcolor{completo}{✓}] Corrección por población finita (1 - n/N)
    \item[\textcolor{completo}{✓}] Mención de intervalos de confianza
\end{itemize}

\vspace{0.3cm}

\begin{alertblock}{\textcolor{gapmedio}{⚠} Gaps Identificados (25\%)}
\end{alertblock}

\begin{itemize}
    \item[\textcolor{gapurgente}{✗}] \textbf{Falta:} ¿Cómo se incorpora la conglomeración en la varianza?
    \item[\textcolor{gapmedio}{⚠}] \textbf{Falta:} Método de estimación de varianza (linearización, bootstrap, jackknife?)
    \item[\textcolor{gapmedio}{⚠}] \textbf{Falta:} ¿Cómo se calculan grados de libertad?
\end{itemize}
\end{frame}

\begin{frame}{3.3 Ponderadores - Análisis Detallado}
\begin{block}{\textcolor{completo}{✓} Información Presente (90\%)}
\end{block}

\begin{itemize}
    \item[\textcolor{completo}{✓}] Factor de expansión \texttt{expr} claramente identificado
    \item[\textcolor{completo}{✓}] Justificación: corrección de probabilidades desiguales
    \item[\textcolor{completo}{✓}] Integración en estimador de Horvitz-Thompson
    \item[\textcolor{completo}{✓}] Uso en análisis de regresión mencionado
    \item[\textcolor{completo}{✓}] Alertblock destacando importancia de ponderadores
\end{itemize}

\vspace{0.3cm}

\begin{alertblock}{\textcolor{gapmedio}{⚠} Gaps Menores (10\%)}
\end{alertblock}

\begin{itemize}
    \item[\textcolor{gapmedio}{⚠}] \textbf{Falta:} ¿Los ponderadores incluyen ajustes por no respuesta?
    \item[\textcolor{gapmedio}{⚠}] \textbf{Falta:} Rango de valores del factor de expansión
\end{itemize}
\end{frame}

\begin{frame}{3.4 Software - Análisis Detallado}
\begin{block}{\textcolor{completo}{✓} Información Presente (95\%)}
\end{block}

\textbf{Paquetes de R:}
\begin{itemize}
    \item[\textcolor{completo}{✓}] \texttt{survey} y \texttt{srvyr} - diseño complejo
    \item[\textcolor{completo}{✓}] \texttt{ggplot2} - visualización
    \item[\textcolor{completo}{✓}] \texttt{dplyr} - procesamiento
\end{itemize}

\textbf{Paquetes de Python:}
\begin{itemize}
    \item[\textcolor{completo}{✓}] \texttt{pandas} - manipulación
    \item[\textcolor{completo}{✓}] \texttt{numpy} - cálculos numéricos
    \item[\textcolor{completo}{✓}] \texttt{matplotlib} y \texttt{seaborn} - visualización
    \item[\textcolor{completo}{✓}] \texttt{statsmodels} - análisis estadístico
\end{itemize}

\vspace{0.2cm}

\begin{alertblock}{\textcolor{gapmedio}{⚠} Gap Menor (5\%)}
\end{alertblock}

\begin{itemize}
    \item[\textcolor{gapmedio}{⚠}] \textbf{Falta:} Versiones específicas de software/paquetes
\end{itemize}
\end{frame}

\begin{frame}{3.5 Limitaciones - Análisis Detallado}
\begin{alertblock}{\textcolor{gapurgente}{✗✗✗ GAP CRÍTICO (0\% de información)}}
\textbf{NO HAY NINGUNA DIAPOSITIVA SOBRE LIMITACIONES}
\end{alertblock}

\vspace{0.3cm}

\begin{block}{Limitaciones que DEBEN ser agregadas:}
\end{block}

\begin{enumerate}
    \item[\textcolor{gapurgente}{✗}] \textbf{Limitaciones del diseño muestral:}
    \begin{itemize}
        \item Exclusión de áreas especiales
        \item No inferencia a nivel comunal en todos los casos
        \item Efecto del diseño aumenta varianza vs. MAS
    \end{itemize}
    
    \item[\textcolor{gapurgente}{✗}] \textbf{Limitaciones de los datos:}
    \begin{itemize}
        \item Sesgo de no respuesta
        \item Datos autoreportados (subjetividad)
        \item Missing values en variables clave
    \end{itemize}
\end{enumerate}
\end{frame}

\begin{frame}{3.5 Limitaciones - Análisis Detallado (cont.)}
\begin{block}{Limitaciones que DEBEN ser agregadas (continuación):}
\end{block}

\begin{enumerate}
    \setcounter{enumi}{2}
    \item[\textcolor{gapurgente}{✗}] \textbf{Limitaciones metodológicas:}
    \begin{itemize}
        \item Causalidad vs. asociación en análisis de regresión
        \item Variables confundidoras no incluidas
        \item Supuestos de los modelos estadísticos
    \end{itemize}
    
    \item[\textcolor{gapurgente}{✗}] \textbf{Limitaciones de inferencia:}
    \begin{itemize}
        \item Representatividad solo para viviendas particulares
        \item Datos de corte transversal (no se puede analizar cambios temporales)
        \item Intervalos de confianza pueden ser amplios en subgrupos pequeños
    \end{itemize}
\end{enumerate}

\vspace{0.3cm}

\begin{alertblock}{Acción Requerida}
\textbf{CREAR UNA O DOS DIAPOSITIVAS DE LIMITACIONES}
\end{alertblock}
\end{frame}

% ============================================================================
% SECCIÓN 4: CRITERIOS DE EVALUACIÓN
% ============================================================================
\section{Criterios de Evaluación}

\begin{frame}{4. Criterios de Evaluación - Estado General}
\begin{block}{Resumen de Completitud}
Evaluación contra criterios del curso
\end{block}

\vspace{0.3cm}

\begin{tabular}{l c l}
\toprule
\textbf{Criterio} & \textbf{Estado} & \textbf{Evaluación} \\
\midrule
Claridad Exposición Oral & \textcolor{completo}{✓} & Bueno \\
Comprensión Diseño & \textcolor{completo}{✓} & Muy Bueno \\
Rigor Metodológico & \textcolor{gapmedio}{⚠} & Regular \\
Calidad Diapositivas & \textcolor{completo}{✓} & Muy Bueno \\
\bottomrule
\end{tabular}

\vspace{0.3cm}

\begin{alertblock}{Evaluación General}
\textbf{Bueno con reservas} - Rigor metodológico afectado por falta de limitaciones
\end{alertblock}
\end{frame}

\begin{frame}{4.1 Claridad en Exposición Oral}
\begin{block}{\textcolor{completo}{✓} Fortalezas}
\end{block}

\begin{itemize}
    \item[\textcolor{completo}{✓}] Texto conciso en diapositivas
    \item[\textcolor{completo}{✓}] Puntos clave bien resaltados con colores
    \item[\textcolor{completo}{✓}] Uso de bloques para organizar información
    \item[\textcolor{completo}{✓}] Buena estructura con secciones claras
    \item[\textcolor{completo}{✓}] Transiciones lógicas entre temas
\end{itemize}

\vspace{0.3cm}

\begin{alertblock}{\textcolor{gapmedio}{⚠} Áreas de Mejora}
\end{alertblock}

\begin{itemize}
    \item[\textcolor{gapmedio}{⚠}] Algunas diapositivas tienen mucho texto (especialmente metodología)
    \item[\textcolor{gapmedio}{⚠}] Falta estimación de tiempo por sección
    \item[\textcolor{gapmedio}{⚠}] Considerar agregar más ejemplos visuales
\end{itemize}
\end{frame}

\begin{frame}{4.2 Comprensión del Diseño Muestral}
\begin{block}{\textcolor{completo}{✓} Fortalezas}
\end{block}

\begin{itemize}
    \item[\textcolor{completo}{✓}] Explicación clara de diseño bietápico
    \item[\textcolor{completo}{✓}] Buena descripción de estratificación
    \item[\textcolor{completo}{✓}] Conexión entre marco muestral y UPM
    \item[\textcolor{completo}{✓}] Presentación de tamaño de muestra con tabla
    \item[\textcolor{completo}{✓}] Mención de métodos de selección (PPT, MAS)
\end{itemize}

\vspace{0.3cm}

\begin{alertblock}{\textcolor{gapmedio}{⚠} Áreas de Mejora}
\end{alertblock}

\begin{itemize}
    \item[\textcolor{gapmedio}{⚠}] Falta justificación de elecciones de diseño
    \item[\textcolor{gapmedio}{⚠}] No se explicita nivel de inferencia específico
    \item[\textcolor{gapmedio}{⚠}] Falta discusión de implicaciones del diseño para análisis
\end{itemize}
\end{frame}

\begin{frame}{4.3 Rigurosidad y Coherencia Metodológica}
\begin{block}{\textcolor{completo}{✓} Fortalezas}
\end{block}

\begin{itemize}
    \item[\textcolor{completo}{✓}] Estimador de Horvitz-Thompson bien presentado
    \item[\textcolor{completo}{✓}] Uso apropiado de ponderadores
    \item[\textcolor{completo}{✓}] Métodos apropiados para objetivos
    \item[\textcolor{completo}{✓}] Coherencia entre objetivos y variables
    \item[\textcolor{completo}{✓}] Software apropiado identificado
\end{itemize}

\vspace{0.3cm}

\begin{alertblock}{\textcolor{gapurgente}{✗} Debilidades Importantes}
\end{alertblock}

\begin{itemize}
    \item[\textcolor{gapurgente}{✗}] \textbf{FALTA COMPLETAMENTE: Sección de limitaciones}
    \item[\textcolor{gapurgente}{✗}] Falta explicación de cómo se incorpora diseño complejo en regresiones
    \item[\textcolor{gapmedio}{⚠}] Falta detalles sobre estimación de varianza con conglomeración
    \item[\textcolor{gapmedio}{⚠}] No se discuten supuestos de los modelos
\end{itemize}
\end{frame}

\begin{frame}{4.4 Calidad de las Diapositivas}
\begin{block}{\textcolor{completo}{✓} Fortalezas}
\end{block}

\begin{itemize}
    \item[\textcolor{completo}{✓}] Diseño consistente con colores institucionales PUC
    \item[\textcolor{completo}{✓}] Excelente legibilidad de texto
    \item[\textcolor{completo}{✓}] Buen uso de ecuaciones matemáticas
    \item[\textcolor{completo}{✓}] Tablas bien formateadas con colores
    \item[\textcolor{completo}{✓}] Balance apropiado entre texto y espacio blanco
    \item[\textcolor{completo}{✓}] Uso efectivo de bloques de color
    \item[\textcolor{completo}{✓}] Logo institucional bien posicionado en portada
\end{itemize}

\vspace{0.3cm}

\begin{alertblock}{\textcolor{gapmedio}{⚠} Mejora Menor}
\end{alertblock}

\begin{itemize}
    \item[\textcolor{gapmedio}{⚠}] Considerar agregar más gráficos/diagramas visuales
\end{itemize}
\end{frame}

% ============================================================================
% RESUMEN EJECUTIVO
% ============================================================================
\section{Resumen y Recomendaciones}

\begin{frame}{Resumen Ejecutivo - Completitud por Sección}
\begin{table}
\centering
\footnotesize
\begin{tabular}{l c c}
\toprule
\textbf{Sección Principal} & \textbf{Completitud} & \textbf{Estado} \\
\midrule
\textbf{1. Diseño Muestral} & 80\% & \textcolor{gapmedio}{⚠} Regular \\
\quad 1.1 Tipo de Diseño & 95\% & \textcolor{completo}{✓} Excelente \\
\quad 1.2 Tamaño de Muestra & 70\% & \textcolor{gapmedio}{⚠} Regular \\
\quad 1.3 Marco Muestral & 90\% & \textcolor{completo}{✓} Muy Bueno \\
\quad 1.4 Nivel de Inferencia & 65\% & \textcolor{gapmedio}{⚠} Regular \\
\midrule
\textbf{2. Plan de Análisis} & 82\% & \textcolor{gapmedio}{⚠} Regular \\
\quad 2.1 Objetivos & 85\% & \textcolor{completo}{✓} Muy Bueno \\
\quad 2.2 Variables & 90\% & \textcolor{completo}{✓} Muy Bueno \\
\quad 2.3 Métodos & 70\% & \textcolor{gapmedio}{⚠} Regular \\
\midrule
\textbf{3. Metodología Detallada} & 69\% & \textcolor{gapurgente}{✗} Deficiente \\
\quad 3.1 Estimadores & 85\% & \textcolor{completo}{✓} Muy Bueno \\
\quad 3.2 Varianzas & 75\% & \textcolor{gapmedio}{⚠} Regular \\
\quad 3.3 Ponderadores & 90\% & \textcolor{completo}{✓} Muy Bueno \\
\quad 3.4 Software & 95\% & \textcolor{completo}{✓} Excelente \\
\quad 3.5 Limitaciones & \textcolor{gapurgente}{0\%} & \textcolor{gapurgente}{✗✗✗} Ausente \\
\bottomrule
\end{tabular}
\end{table}

\vspace{0.2cm}

\begin{alertblock}{Completitud Global}
\textbf{77\%} - Presentación necesita ajustes antes de entrega final
\end{alertblock}
\end{frame}

\begin{frame}{Prioridades de Acción}
\begin{block}{\textcolor{gapurgente}{PRIORIDAD CRÍTICA} (Requiere atención inmediata)}
\end{block}

\begin{enumerate}
    \item[\textcolor{gapurgente}{1.}] \textbf{CREAR DIAPOSITIVA(S) DE LIMITACIONES}
    \begin{itemize}
        \item Del diseño muestral
        \item De los datos
        \item Metodológicas
        \item De inferencia
    \end{itemize}
    
    \item[\textcolor{gapurgente}{2.}] \textbf{Especificar niveles de inferencia exactos}
    \begin{itemize}
        \item Nacional, regional, zona urbana/rural
        \item Limitaciones por tamaño de muestra
    \end{itemize}
    
    \item[\textcolor{gapurgente}{3.}] \textbf{Explicar incorporación de diseño complejo en análisis}
    \begin{itemize}
        \item ¿Cómo se usa \texttt{survey} en regresiones?
        \item Estimación de errores estándar robustos
    \end{itemize}
\end{enumerate}
\end{frame}

\begin{frame}{Prioridades de Acción (continuación)}
\begin{block}{\textcolor{gapmedio}{PRIORIDAD ALTA} (Debe completarse antes de presentación)}
\end{block}

\begin{enumerate}
    \setcounter{enumi}{3}
    \item[\textcolor{gapmedio}{4.}] Agregar explicación de cálculo de errores muestrales
    
    \item[\textcolor{gapmedio}{5.}] Incluir tasa de respuesta y su impacto
    
    \item[\textcolor{gapmedio}{6.}] Detallar método de estimación de varianza con conglomeración
    
    \item[\textcolor{gapmedio}{7.}] Especificar nivel de significancia y manejo de missing values
\end{enumerate}

\vspace{0.3cm}

\begin{block}{\textcolor{gapbajo}{PRIORIDAD MEDIA} (Mejoras recomendadas)}
\end{block}

\begin{enumerate}
    \setcounter{enumi}{7}
    \item[\textcolor{gapbajo}{8.}] Agregar justificación de elección de diseño bietápico
    
    \item[\textcolor{gapbajo}{9.}] Incluir escalas de medición de variables
    
    \item[\textcolor{gapbajo}{10.}] Agregar más visualizaciones (gráficos, diagramas)
\end{enumerate}
\end{frame}

\begin{frame}{Recomendaciones Finales}
\begin{block}{Para mejorar la calidad de la presentación:}
\end{block}

\begin{enumerate}
    \item \textbf{Agregar 1-2 diapositivas de Limitaciones} después de Software
    \begin{itemize}
        \item Es un requisito explícito de contenidos mínimos
        \item Demuestra comprensión profunda del análisis
        \item Aumenta credibilidad académica
    \end{itemize}
    
    \item \textbf{Mejorar conexiones explícitas} entre secciones
    \begin{itemize}
        \item "Debido al diseño bietápico, usaremos..."
        \item "Para incorporar la estratificación, el software..."
    \end{itemize}
    
    \item \textbf{Preparar respuestas a preguntas potenciales}
    \begin{itemize}
        \item ¿Por qué Horvitz-Thompson y no otro estimador?
        \item ¿Cómo afecta la conglomeración a la precisión?
        \item ¿Qué limitaciones tiene el marco muestral?
    \end{itemize}
\end{enumerate}
\end{frame}

\begin{frame}{Cronograma Sugerido}
\begin{table}
\centering
\footnotesize
\begin{tabular}{l l l}
\toprule
\textbf{Prioridad} & \textbf{Tarea} & \textbf{Tiempo Estimado} \\
\midrule
\textcolor{gapurgente}{Crítica} & Crear diapositivas de limitaciones & 2-3 horas \\
\textcolor{gapurgente}{Crítica} & Especificar niveles de inferencia & 1 hora \\
\textcolor{gapurgente}{Crítica} & Explicar diseño complejo en análisis & 1-2 horas \\
\midrule
\textcolor{gapmedio}{Alta} & Agregar cálculo de errores & 1 hora \\
\textcolor{gapmedio}{Alta} & Incluir tasa de respuesta & 30 min \\
\textcolor{gapmedio}{Alta} & Detallar varianza con conglomeración & 1 hora \\
\textcolor{gapmedio}{Alta} & Nivel de significancia y missing values & 30 min \\
\midrule
\textcolor{gapbajo}{Media} & Justificación de diseño & 1 hora \\
\textcolor{gapbajo}{Media} & Escalas de medición & 30 min \\
\textcolor{gapbajo}{Media} & Agregar visualizaciones & 2 horas \\
\bottomrule
\end{tabular}
\end{table}

\vspace{0.2cm}

\begin{alertblock}{Tiempo Total Estimado}
\textbf{11-13 horas} para completar todos los gaps
\end{alertblock}
\end{frame}

\begin{frame}{Conclusión}
\begin{block}{Estado Actual}
La presentación tiene una \textbf{base sólida} (77\% de completitud) con:
\end{block}

\begin{itemize}
    \item[\textcolor{completo}{+}] Excelente calidad visual
    \item[\textcolor{completo}{+}] Buena explicación del diseño muestral
    \item[\textcolor{completo}{+}] Software bien identificado
    \item[\textcolor{completo}{+}] Objetivos claros
\end{itemize}

\vspace{0.3cm}

\begin{alertblock}{Gap Crítico}
\textbf{FALTA SECCIÓN DE LIMITACIONES} - Requisito obligatorio
\end{alertblock}

\vspace{0.3cm}

\begin{block}{Con los ajustes sugeridos:}
Se puede alcanzar \textbf{90-95\% de completitud} y obtener una excelente evaluación
\end{block}
\end{frame}

% ============================================================================
% CIERRE
% ============================================================================
\begin{frame}[plain]
\begin{center}
\vspace{2cm}
{\huge \textcolor{celesteoscuro}{\textbf{¿Preguntas?}}}

\vspace{1cm}

{\large Grupo 4}

\vspace{0.5cm}

{\normalsize Alexander Pinto, Esteban Román,}\\
{\normalsize Julián Vargas, Francisca Sepúlveda}

\vspace{1cm}

{\small \textcolor{grisoscuro}{Este análisis de gaps fue generado automáticamente}}\\
{\small \textcolor{grisoscuro}{basado en la verificación de contenidos mínimos}}
\end{center}
\end{frame}

\end{document}
