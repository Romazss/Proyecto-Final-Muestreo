% ============================================================================
% PLAN DE ANÁLISIS
% ============================================================================
\section{Plan de Análisis}

\begin{frame}{Objetivo 1: Brecha Salarial de Género}
\begin{block}{Objetivo}
Analizar diferencias por género en variables socioeconómicas, laborales y educativas utilizando CASEN 2022
\end{block}

\vspace{0.3cm}

\begin{columns}[t]
\begin{column}{0.48\textwidth}
\textcolor{celesteoscuro}{\textbf{Variables principales:}}
\begin{itemize}
    \item \texttt{sexo}: Sexo de la persona
    \item \texttt{ytrabajocorh}: Ingreso del trabajo principal corregido
    \item \texttt{esc}: Años de educación formal
\end{itemize}
\end{column}

\begin{column}{0.48\textwidth}
\textcolor{celesteoscuro}{\textbf{Variables de control:}}
\begin{itemize}
    \item \texttt{edad}: Edad de la persona
    \item \texttt{oficio4\_08}: Ocupación
    \item \texttt{tot\_per\_h}: Total personas en el hogar
\end{itemize}
\end{column}
\end{columns}
\end{frame}

\begin{frame}{Metodología - Objetivo 1}
\begin{block}{Métodos Propuestos}
Análisis de la brecha salarial de género
\end{block}

\vspace{0.4cm}

\begin{enumerate}
    \item[\textcolor{celesteprincipal}{\textbf{1.}}] \textbf{Análisis descriptivo:} Cálculo de ingresos promedio del trabajo principal según sexo
    
    \vspace{0.3cm}
    
    \item[\textcolor{celesteprincipal}{\textbf{2.}}] \textbf{Modelo de regresión lineal ponderado:}
    \begin{itemize}
        \item Variable dependiente: \texttt{ytrabajocorh} (Ingreso)
        \item Variables explicativas: sexo, educación, edad y ocupación
        \item Utiliza factores de expansión del diseño muestral
    \end{itemize}
\end{enumerate}
\end{frame}

\begin{frame}{Objetivo 2: Distribución de la Pobreza}
\begin{block}{Objetivo}
Explorar la distribución de la pobreza y diferencias entre zonas rural y urbana
\end{block}

\vspace{0.3cm}

\begin{columns}[t]
\begin{column}{0.48\textwidth}
\textcolor{celesteoscuro}{\textbf{Variables principales:}}
\begin{itemize}
    \item \texttt{pobreza}: Condición de pobreza
    \item \texttt{ytotcorh}: Ingreso total corregido
    \item \texttt{zona}: Rural/Urbana
\end{itemize}
\end{column}

\begin{column}{0.48\textwidth}
\textcolor{celesteoscuro}{\textbf{Variables de control:}}
\begin{itemize}
    \item \texttt{esc}: Años de educación
    \item \texttt{región}: Región del país
    \item \texttt{edad}: Edad de la persona
    \item \texttt{expr}: Factor de expansión
\end{itemize}
\end{column}
\end{columns}
\end{frame}

\begin{frame}{Metodología - Objetivo 2}
\begin{block}{Métodos Propuestos}
Análisis multidimensional de la pobreza
\end{block}

\vspace{0.3cm}

\begin{enumerate}
    \item[\textcolor{celesteprincipal}{\textbf{1.}}] Calcular el \textbf{porcentaje de hogares en pobreza} por región y zona (rural/urbana)
    
    \item[\textcolor{celesteprincipal}{\textbf{2.}}] Comparar el \textbf{ingreso promedio} según nivel educativo y tamaño del hogar
    
    \item[\textcolor{celesteprincipal}{\textbf{3.}}] \textbf{Pruebas de hipótesis ponderadas} para determinar si las diferencias entre zonas son estadísticamente significativas
    
    \item[\textcolor{celesteprincipal}{\textbf{4.}}] \textbf{Modelo de regresión logística} para explorar condiciones de pobreza con variables: educación, tamaño del hogar, edad y zona
\end{enumerate}
\end{frame}
