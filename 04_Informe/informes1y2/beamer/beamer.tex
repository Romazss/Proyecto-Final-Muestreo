% ============================================================================
% PRESENTACIÓN BEAMER - Proyecto Final Muestreo
% ============================================================================
% Presentación profesional con colores institucionales PUC
% Tema visual coherente con el informe principal

\documentclass[11pt,aspectratio=169]{beamer}

% ============================================================================
% TEMA Y CONFIGURACIÓN VISUAL
% ============================================================================
\usetheme{Madrid}
\usecolortheme{default}
\usefonttheme{serif}

% ============================================================================
% PAQUETES
% ============================================================================
\usepackage[utf8]{inputenc}
\usepackage[spanish]{babel}
\usepackage[T1]{fontenc}
\usepackage{multirow}
\usepackage{booktabs}
\usepackage{amsmath}
\usepackage{amsfonts}
\usepackage{amssymb}
\usepackage{graphicx}
\usepackage{tikz}
\usepackage{xcolor}
\usepackage{colortbl}

% ============================================================================
% DEFINICIÓN DE COLORES PUC (Coherente con el informe)
% ============================================================================
\definecolor{celesteprincipal}{RGB}{0,150,200}
\definecolor{celestesuave}{RGB}{135,206,235}
\definecolor{celesteclaro}{RGB}{173,216,230}
\definecolor{celesteoscuro}{RGB}{0,105,148}
\definecolor{celestefondo}{RGB}{230,245,255}
\definecolor{grisoscuro}{RGB}{64,64,64}

% ============================================================================
% CONFIGURACIÓN DE COLORES DEL TEMA
% ============================================================================
\setbeamercolor{structure}{fg=celesteoscuro}
\setbeamercolor{palette primary}{bg=celesteprincipal,fg=white}
\setbeamercolor{palette secondary}{bg=celestesuave,fg=white}
\setbeamercolor{palette tertiary}{bg=celesteoscuro,fg=white}
\setbeamercolor{palette quaternary}{bg=celesteclaro,fg=grisoscuro}

\setbeamercolor{titlelike}{parent=palette primary}
\setbeamercolor{frametitle}{bg=celesteprincipal,fg=white}
\setbeamercolor{block title}{bg=celesteprincipal,fg=white}
\setbeamercolor{block body}{bg=celestefondo,fg=grisoscuro}
\setbeamercolor{item}{fg=celesteoscuro}
\setbeamercolor{section in toc}{fg=celesteoscuro}

% ============================================================================
% CONFIGURACIÓN DE PLANTILLAS
% ============================================================================
\setbeamertemplate{caption}[numbered]
\setbeamertemplate{navigation symbols}{}
\setbeamertemplate{footline}[frame number]
\setbeamertemplate{itemize items}[circle]
\setbeamertemplate{enumerate items}[default]

% Configuración de bloques con bordes redondeados
\setbeamertemplate{blocks}[rounded][shadow=true]

% ============================================================================
% INFORMACIÓN DEL DOCUMENTO
% ============================================================================
\title[Métodos - CASEN 2022]{Análisis de Desigualdades Socioeconómicas}
\subtitle{Encuesta CASEN 2022 - Métodos y Diseño Muestral}
\author[Grupo 4]{
    Alexander Pinto \and 
    Esteban Román \and 
    Julián Vargas \and 
    Francisca Sepúlveda
}
\institute[PUC]{
    \Large Pontificia Universidad Católica de Chile \\[0.3cm]
    \small EYP2417 - Muestreo \\
    \small Grupo 4
}
\date{11 de noviembre de 2025}

% Logo en el pie de página
\logo{\includegraphics[height=0.8cm]{../03_Logos/logo_kovan.jpg}}

% ============================================================================
% CONFIGURACIÓN DE BIBLIOGRAFÍA
% ============================================================================
\bibliographystyle{apalike}

% ============================================================================
% INICIO DEL DOCUMENTO
% ============================================================================
\begin{document}

% ============================================================================
% PORTADA
% ============================================================================
{
\setbeamertemplate{footline}{} % Quitar número de página en portada
\begin{frame}[plain]
\titlepage
\end{frame}
}

% ============================================================================
% TABLA DE CONTENIDOS
% ============================================================================
\begin{frame}{Contenido}
\tableofcontents[hideallsubsections]
\end{frame}

% ============================================================================
% INTRODUCCIÓN
% ============================================================================
\section{Introducción}

\begin{frame}{Introducción}
\begin{block}{Encuesta CASEN}
La Encuesta de Caracterización Socioeconómica Nacional (CASEN) tiene como objetivo medir las condiciones de vida de los hogares y la población en el territorio chileno.
\end{block}

\vspace{0.3cm}

\begin{itemize}
    \item[\checkmark] Permite estimar \textbf{indicadores de pobreza}, desigualdad e inclusión social a nivel nacional
    \item[\checkmark] Representa a la población que reside en \textcolor{celesteoscuro}{\textbf{viviendas particulares ocupadas}} en todo el territorio nacional
    \item[\checkmark] Excluye las denominadas \textcolor{celesteoscuro}{\textbf{áreas especiales}} (zonas de acceso restringido, alto costo o condiciones climáticas adversas)
\end{itemize}
\end{frame}

% ============================================================================
% DISEÑO MUESTRAL
% ============================================================================
\section{Diseño Muestral}

\begin{frame}{Diseño Muestral}
\begin{block}{Características principales}
El diseño muestral de la Encuesta CASEN se caracteriza por ser:
\end{block}

\vspace{0.5cm}

\begin{enumerate}
    \item[\textcolor{celesteprincipal}{\textbf{1.}}] \textbf{Probabilístico:} Todas las unidades de la población tienen una probabilidad conocida y distinta de cero de ser seleccionadas
    
    \vspace{0.3cm}
    
    \item[\textcolor{celesteprincipal}{\textbf{2.}}] \textbf{Estratificado:} El territorio se divide en \textcolor{celesteoscuro}{\textbf{estratos geográficos y socioeconómicos}}
    
    \begin{itemize}
        \item Mejora la precisión de las estimaciones
        \item Asegura representatividad en distintos niveles (nacional, regional y urbano/rural)
    \end{itemize}
\end{enumerate}
\end{frame}

\begin{frame}{Diseño Muestral}
\begin{enumerate}
    \setcounter{enumi}{2}
    \item[\textcolor{celesteprincipal}{\textbf{3.}}] \textbf{Bietápico:} La selección se realiza en dos etapas:
    
    \vspace{0.4cm}
    
    \begin{block}{Primera etapa}
    Selección \textbf{sistemática con probabilidad proporcional al tamaño (PPT)} de las \textcolor{celesteoscuro}{\textbf{unidades primarias de muestreo (UPM)}} - conglomerados de viviendas
    \end{block}
    
    \vspace{0.3cm}
    
    \begin{block}{Segunda etapa}
    Selección \textbf{aleatoria simple (MAS)} de \textcolor{celesteoscuro}{\textbf{viviendas}} dentro de cada unidad primaria
    \end{block}
\end{enumerate}
\end{frame}    

\begin{frame}{Tamaño Muestral}
\begin{block}{Consideraciones para el tamaño de muestra óptimo}
CASEN considera los siguientes criterios:
\end{block}

\vspace{0.3cm}

\begin{enumerate}
    \item[\textcolor{celesteprincipal}{\textbf{1.}}] Representatividad del \textbf{territorio nacional}, regiones y zonas urbanas/rurales
    
    \vspace{0.2cm}
    
    \item[\textcolor{celesteprincipal}{\textbf{2.}}] \textbf{Simulaciones} para optimizar precisión
\end{enumerate}     
\end{frame}

\begin{frame}{Tamaño de Muestra CASEN 2022}
\begin{table}[h]
\centering
\footnotesize
\renewcommand{\arraystretch}{1.3}
\begin{tabular}{>{\columncolor{celestefondo}}l l c c c c}
\toprule
\rowcolor{celesteprincipal}
\textcolor{white}{\textbf{}} & \textcolor{white}{\textbf{Nivel}}  & \textcolor{white}{\textbf{Tamaño}} & \textcolor{white}{\textbf{Error}} & \textcolor{white}{\textbf{Error}} & \textcolor{white}{\textbf{Tamaño con}} \\
\rowcolor{celesteprincipal}
\textcolor{white}{\textbf{}} & \textcolor{white}{\textbf{}}  & \textcolor{white}{\textbf{Objetivo}} & \textcolor{white}{\textbf{Absoluto}} & \textcolor{white}{\textbf{Relativo}} & \textcolor{white}{\textbf{Sobremuestreo}} \\
\midrule
\rowcolor{white}
 & \textbf{País}  & 71.028 & 0,4\% & 3,3\% & 106.856 \\
\rowcolor{celestefondo}
 & Urbano & 56.905 & 0,5\% & 4,6\% & 87.252  \\
\rowcolor{white}
 & Rural  & 14.123 & 1,3\% & 9,2\% & 19.604  \\
\bottomrule
\end{tabular}
\end{table}

\vspace{0.2cm}
\small
\textcolor{grisoscuro}{\textit{Fuente: Manual Metodológico CASEN 2022, p. 35}}
\end{frame}

\begin{frame}{Marco Muestral}
\begin{block}{Construcción del Marco}
A partir del \textcolor{celesteoscuro}{\textbf{MMV 2020}} se elabora el marco de selección para CASEN 2022
\end{block}

\vspace{0.3cm}

\begin{itemize}
    \item Conformado por \textbf{335 comunas} definidas para el nuevo diseño de la Encuesta Nacional de Empleo (ENE 2020)
    
    \vspace{0.2cm}
    
    \item Las UPM del marco muestral están estratificadas por:
    \begin{enumerate}
        \item[\textcolor{celesteprincipal}{$\bullet$}] \textbf{Geografía} (comunas)
        \item[\textcolor{celesteprincipal}{$\bullet$}] \textbf{Áreas} (urbana–rural)
        \item[\textcolor{celesteprincipal}{$\bullet$}] \textbf{Nivel socioeconómico (NSE)} - construida a partir de la clasificación del MMV 2020
    \end{enumerate}
\end{itemize}
\end{frame}

\begin{frame}{Nivel de Inferencia}
\begin{block}{Objetivo Analítico}
Producir inferencias hacia la población que reside en \textcolor{celesteoscuro}{\textbf{viviendas particulares ocupadas}} (elegibles)
\end{block}

\vspace{0.4cm}

\begin{alertblock}{Exclusiones}
No se consideran para fines analíticos las viviendas \textbf{no elegibles}:
\end{alertblock}

\vspace{0.2cm}

\begin{columns}[t]
\begin{column}{0.48\textwidth}
\begin{itemize}
    \item Oficinas de empresas
    \item Viviendas abandonadas
\end{itemize}
\end{column}
\begin{column}{0.48\textwidth}
\begin{itemize}
    \item Viviendas de veraneo
    \item Viviendas demolidas
\end{itemize}
\end{column}
\end{columns}
\end{frame}

% ============================================================================
% PLAN DE ANÁLISIS
% ============================================================================
\section{Plan de Análisis}

\begin{frame}{Objetivo 1: Brecha Salarial de Género}
\begin{block}{Objetivo}
Analizar diferencias por género en variables socioeconómicas, laborales y educativas utilizando CASEN 2022
\end{block}

\vspace{0.3cm}

\begin{columns}[t]
\begin{column}{0.48\textwidth}
\textcolor{celesteoscuro}{\textbf{Variables principales:}}
\begin{itemize}
    \item \texttt{sexo}: Sexo de la persona
    \item \texttt{ytrabajocorh}: Ingreso del trabajo principal corregido
    \item \texttt{esc}: Años de educación formal
\end{itemize}
\end{column}

\begin{column}{0.48\textwidth}
\textcolor{celesteoscuro}{\textbf{Variables de control:}}
\begin{itemize}
    \item \texttt{edad}: Edad de la persona
    \item \texttt{oficio4\_08}: Ocupación
    \item \texttt{tot\_per\_h}: Total personas en el hogar
\end{itemize}
\end{column}
\end{columns}
\end{frame}

\begin{frame}{Metodología - Objetivo 1}
\begin{block}{Métodos Propuestos}
Análisis de la brecha salarial de género
\end{block}

\vspace{0.4cm}

\begin{enumerate}
    \item[\textcolor{celesteprincipal}{\textbf{1.}}] \textbf{Análisis descriptivo:} Cálculo de ingresos promedio del trabajo principal según sexo
    
    \vspace{0.3cm}
    
    \item[\textcolor{celesteprincipal}{\textbf{2.}}] \textbf{Modelo de regresión lineal ponderado:}
    \begin{itemize}
        \item Variable dependiente: \texttt{ytrabajocorh} (Ingreso)
        \item Variables explicativas: sexo, educación, edad y ocupación
        \item Utiliza factores de expansión del diseño muestral
    \end{itemize}
\end{enumerate}
\end{frame}
    


\begin{frame}{Objetivo 2: Distribución de la Pobreza}
\begin{block}{Objetivo}
Explorar la distribución de la pobreza y diferencias entre zonas rural y urbana
\end{block}

\vspace{0.3cm}

\begin{columns}[t]
\begin{column}{0.48\textwidth}
\textcolor{celesteoscuro}{\textbf{Variables principales:}}
\begin{itemize}
    \item \texttt{pobreza}: Condición de pobreza
    \item \texttt{ytotcorh}: Ingreso total corregido
    \item \texttt{zona}: Rural/Urbana
\end{itemize}
\end{column}

\begin{column}{0.48\textwidth}
\textcolor{celesteoscuro}{\textbf{Variables de control:}}
\begin{itemize}
    \item \texttt{esc}: Años de educación
    \item \texttt{región}: Región del país
    \item \texttt{edad}: Edad de la persona
    \item \texttt{expr}: Factor de expansión
\end{itemize}
\end{column}
\end{columns}
\end{frame}

\begin{frame}{Metodología - Objetivo 2}
\begin{block}{Métodos Propuestos}
Análisis multidimensional de la pobreza
\end{block}

\vspace{0.3cm}

\begin{enumerate}
    \item[\textcolor{celesteprincipal}{\textbf{1.}}] Calcular el \textbf{porcentaje de hogares en pobreza} por región y zona (rural/urbana)
    
    \item[\textcolor{celesteprincipal}{\textbf{2.}}] Comparar el \textbf{ingreso promedio} según nivel educativo y tamaño del hogar
    
    \item[\textcolor{celesteprincipal}{\textbf{3.}}] \textbf{Pruebas de hipótesis ponderadas} para determinar si las diferencias entre zonas son estadísticamente significativas
    
    \item[\textcolor{celesteprincipal}{\textbf{4.}}] \textbf{Modelo de regresión logística} para explorar condiciones de pobreza con variables: educación, tamaño del hogar, edad y zona
\end{enumerate}
\end{frame}





% ============================================================================
% METODOLOGÍA DETALLADA
% ============================================================================
\section{Metodología Detallada}

\begin{frame}{Estimador de Horvitz-Thompson}
\begin{block}{Estimador de la media ponderada}
Para estimar medias y proporciones bajo un diseño muestral complejo, utilizamos el estimador de Horvitz-Thompson ajustado por el factor de expansión:
\end{block}

\vspace{0.3cm}

\begin{equation*}
\textcolor{celesteoscuro}{\hat{\bar{Y}} = \frac{\sum_{i \in s} \frac{y_i}{\pi_i}}{\sum_{i \in s} \frac{1}{\pi_i}} \approx
\frac{\sum_{i \in s} expr_i \, y_i}{\sum_{i \in s} expr_i}}
\end{equation*}

\vspace{0.3cm}

donde:
\begin{itemize}
    \item $y_i$ corresponde a la \textbf{variable de interés}
    \item $expr_i$ es el \textcolor{celesteoscuro}{\textbf{factor de expansión}} provisto por CASEN 2022
    \item $\pi_i$ es la probabilidad de inclusión de la unidad $i$
\end{itemize}
\end{frame}

\begin{frame}{Varianza del Estimador}
\begin{block}{Varianza e intervalos de confianza}
La varianza del estimador se calcula considerando el \textbf{efecto del diseño} (estratificación y conglomeración):
\end{block}

\vspace{0.3cm}

\begin{equation*}
\textcolor{celesteoscuro}{V(\hat{Y}) = \sum_h \left(1 - \frac{n_h}{N_h}\right) \frac{s_h^2}{n_h}}
\end{equation*}

\vspace{0.4cm}

\begin{alertblock}{Ponderadores}
Se usará el factor de expansión \texttt{expr} para corregir las \textbf{probabilidades desiguales de selección} en el diseño muestral
\end{alertblock}
\end{frame}

\begin{frame}{Software y Herramientas}
\begin{block}{Implementación en R}
Los métodos se implementarán utilizando los siguientes paquetes:
\end{block}

\vspace{0.4cm}

\begin{columns}[t]
\begin{column}{0.45\textwidth}
\begin{itemize}
    \item[\textcolor{celesteprincipal}{$\checkmark$}] \texttt{survey} y \texttt{srvyr}
    \begin{itemize}
        \item Estimaciones bajo diseño complejo
    \end{itemize}
\end{itemize}
\end{column}

\begin{column}{0.45\textwidth}
\begin{itemize}
    \item[\textcolor{celesteprincipal}{$\checkmark$}] \texttt{ggplot2}
    \begin{itemize}
        \item Visualización de datos
    \end{itemize}
    \item[\textcolor{celesteprincipal}{$\checkmark$}] \texttt{dplyr}
    \begin{itemize}
        \item Procesamiento de datos
    \end{itemize}
\end{itemize}
\end{column}
\end{columns}
\end{frame}

% ============================================================================
% CIERRE
% ============================================================================
\begin{frame}[plain]
\begin{center}
\vspace{2cm}
{\Huge \textcolor{celesteoscuro}{\textbf{¿Preguntas?}}}

\vspace{1.5cm}

{\Large Gracias por su atención}

\vspace{1cm}

\textcolor{grisoscuro}{
Alexander Pinto | Esteban Román \\
Julián Vargas | Francisca Sepúlveda
}

\vspace{0.5cm}

\textcolor{celesteprincipal}{\textbf{Pontificia Universidad Católica de Chile}}
\end{center}
\end{frame}
    
\end{document}