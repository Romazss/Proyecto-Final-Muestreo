% ============================================================================
% 01_Estructura/04_revision_antecedentes.tex
% ============================================================================

\section{Revisión de Antecedentes}

\subsection{Estudios sobre Pobreza en Chile}

La pobreza en Chile ha sido objeto de múltiples estudios. A continuación se resumen investigaciones previas relevantes:

\subsubsection*{Reportes Oficiales CASEN}

El Ministerio de Desarrollo Social y Familia publica anualmente reportes estadísticos basados en CASEN. Los reportes más recientes (2020-2022) documentan:

\begin{itemize}
    \item Evolución de tasas de pobreza por región y zona (urbano/rural)
    \item Descomposición de pobreza por características demográficas
    \item Análisis de vulnerabilidad y pobreza extrema
    \item Comparaciones temporales desde 2015
\end{itemize}

Estos reportes sirven como línea base para validación de nuestros estimadores y como contexto para interpretar resultados.

\subsubsection*{Literatura Académica}

Trabajos recientes en Chile incluyen:

\begin{itemize}
    \item \textbf{Contreras y Núñez (2017):} Análisis de movilidad social y persistencia de pobreza intergeneracional usando datos CASEN
    \item \textbf{Agostini y Brown (2007):} Estimación de elasticidad ingreso-pobreza en zonas urbanas y rurales
    \item \textbf{CEPAL (2021):} Perspectivas socioeconómicas en América Latina y el Caribe, incluida caracterización de pobreza en Chile post-COVID
\end{itemize}

\subsection{Estudios sobre Brecha Salarial de Género}

La brecha de género en ingresos es ampliamente documentada en literatura internacional y nacional:

\subsubsection*{Reportes Nacionales}

\begin{itemize}
    \item \textbf{Instituto Nacional de Estadísticas (INE):} Publica cifras sobre brechas salariales a partir de encuestas de empleo
    \item \textbf{Servicio Nacional de la Mujer (SERNAMEG):} Reportes anuales sobre equidad de género, incluyendo análisis salarial
    \item \textbf{Ministerio del Trabajo:} Diagnósticos sobre igualdad laboral y paridad de género
\end{itemize}

\subsubsection*{Literatura Académica}

Estudios aplicados a Chile:

\begin{itemize}
    \item \textbf{Contreras et al. (1999):} Descomposición de la brecha salarial por género usando encuestas de empleo
    \item \textbf{Aldén y Hammarstedt (2014):} Análisis de brecha salarial en el contexto latinoamericano con datos CASEN
    \item \textbf{Fuentes y Palma (2014):} Estimación de discriminación salarial por género en Chile controlando por educación y experiencia
\end{itemize}

\subsection{Metodología de Muestreo Aplicada}

La correcta especificación del diseño muestral es fundamental. Literatura de referencia:

\begin{itemize}
    \item \textbf{Thompson (2012):} ``Sampling'' - Referencia principal para diseños estratificados polietápicos y estimadores insesgados
    \item \textbf{Lumley (2010, 2011):} Implementación computacional de survey statistics en R
    \item \textbf{Särndal et al. (1992):} ``Model Assisted Survey Sampling'' - Fundamentos teóricos de inference bajo diseños complejos
    \item \textbf{Lohr (2019, 3ª ed.):} ``Sampling: Design and Analysis'' - Aplicaciones prácticas y calibración de ponderadores
\end{itemize}

\subsection{CASEN como Fuente de Datos}

CASEN es ampliamente utilizada en investigación social chilena. Estudios que utilizan CASEN incluyen:

\begin{itemize}
    \item Análisis de desigualdad y distribución de ingresos (Gini, polarización)
    \item Estudios de política social (evaluación de programas, focalización)
    \item Investigaciones sobre educación, empleo y características demográficas
    \item Comparaciones regionales y zonales
\end{itemize}

La base de datos de CASEN 2022 es reciente y pertinente para diagnosticar la situación post-pandemia en Chile.

\subsection{Validez y Pertinencia de Este Análisis}

Este proyecto se justifica por:

\begin{enumerate}
    \item \textbf{Novedad temporal:} CASEN 2022 es reciente y aún no ha sido extensivamente analizada en comparación con rondas anteriores
    \item \textbf{Diseño metodológico riguroso:} Aplicación correcta de Thompson (2012) asegura validez de inferencia
    \item \textbf{Multidimensionalidad:} Análisis simultáneo de pobreza y brecha salarial permite identificar interconexiones
    \item \textbf{Relevancia política:} Contribuye a debate sobre desigualdad socioeconómica y políticas públicas en Chile
\end{enumerate}
