% ============================================================================
% 01_Estructura/introduccion.tex
% ============================================================================

\section{Objetivo General}

\vspace{0.3cm}

La desigualdad socioeconómica es uno de los desafíos estructurales más relevantes de Chile. Este proyecto busca \textbf{contribuir al diagnóstico de la desigualdad chilena} mediante el análisis riguroso de datos de la Encuesta de Caracterización Socioeconómica Nacional (CASEN) 2022.

\vspace{0.5cm}

\begin{hallazgobox}
\noindent\textbf{\Large Dos ejes temáticos complementarios:}

\vspace{0.3cm}

\begin{enumerate}[leftmargin=1.5cm]
    \item[\textcolor{celesteprincipal}{\large\bfseries 1.}] \textbf{Distribución de la pobreza} \\
    Caracterizar geográficamente (por región y zona urbana/rural) y demográficamente la incidencia de pobreza en el país.
    
    \vspace{0.2cm}
    
    \item[\textcolor{celesteprincipal}{\large\bfseries 2.}] \textbf{Brecha salarial de género} \\
    Analizar diferencias en ingresos del trabajo entre jefes de hogar hombres y mujeres.
\end{enumerate}
\end{hallazgobox}

\vspace{0.5cm}

\noindent El análisis se fundamenta en \textbf{teoría de muestreo} (Thompson, 2012) considerando el diseño muestral complejo de CASEN. Aplicamos correctamente el \textbf{factor de expansión} (\texttt{expr}) en todas las estimaciones, proporcionando evidencia para políticas públicas orientadas a reducir inequidades.
