% ============================================================================
% 01_Estructura/referencias.tex
% ============================================================================

\newpage

\section*{Referencias}
\addcontentsline{toc}{section}{Referencias}

\begin{enumerate}[label={[\arabic*]}]

    \item \textbf{Thompson, S. K.} (2012). \textit{Sampling} (3rd ed.). New York: Wiley. \\ 
    Texto principal del curso con fundamentos de diseño muestral multietápico.

    \item \textbf{Lohr, S. L.} (2009). \textit{Sampling: Design and Analysis} (2nd ed.). Boston: Brooks/Cole. \\
    Referencia sobre análisis en encuestas complejas.

    \item \textbf{Lumley, T.} (2010). \textit{Complex Surveys: A Guide to Analysis Using R}. New York: Wiley. \\
    Guía práctica de análisis con datos de encuestas complejas.

    \item \textbf{Särndal, C. E., Swensson, B., \& Wretman, J.} (2013). \textit{Model Assisted Survey Sampling}. New York: Springer. \\
    Fundamentos teóricos avanzados de muestreo.

    \item \textbf{Ministerio de Desarrollo Social y Familia.} (2023). \textit{Manual Metodológico CASEN 2022}. \\
    Descripción técnica del diseño muestral y variables CASEN.

    \item \textbf{Ministerio de Desarrollo Social y Familia.} (2023). \textit{Manual del Investigador CASEN 2022}. \\
    Guía de uso de variables y ponderadores.

    \item \textbf{Observatorio Social.} (2024). \textit{Resultados de la Encuesta CASEN 2022}. \\
    Disponible en: \url{https://www.ministeriodesarrollosocial.gob.cl/casen/}

\end{enumerate}
