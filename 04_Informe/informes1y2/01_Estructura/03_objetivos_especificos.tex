% ============================================================================
% 01_Estructura/03_objetivos_especificos.tex
% ============================================================================

\section{Objetivos Específicos}

Este proyecto busca responder a las siguientes preguntas de investigación concretas:

\subsection{Eje 1: Distribución de la Pobreza}

\begin{enumerate}
    \item \textbf{¿Cuál es la incidencia de pobreza por región en Chile?} Se busca caracterizar la distribución geográfica de la pobreza en 2022, identificando qué regiones presentan mayores tasas de pobreza e incidencia de pobreza extrema.
    
    \item \textbf{¿Existen diferencias significativas entre zonas urbanas y rurales?} Se comparará la incidencia de pobreza en áreas urbanas versus rurales, analizando si el factor geográfico (urbanización) es relevante en la explicación de la pobreza.
    
    \item \textbf{¿Cuáles son los factores asociados a la pobreza?} Se analizarán variables como tamaño del hogar, educación del jefe, edad y composición demográfica para identificar patrones asociados a mayor incidencia de pobreza.
\end{enumerate}

\subsection{Eje 2: Brecha Salarial de Género}

\begin{enumerate}
    \item \textbf{¿Existe brecha salarial entre jefes de hogar hombres y mujeres?} Se comparará el ingreso del trabajo principal entre hombres y mujeres que son jefes de hogar, cuantificando la magnitud de la diferencia.
    
    \item \textbf{¿La brecha varía según nivel educativo?} Se estratificará el análisis por años de educación para determinar si la brecha persiste o se amplía conforme aumenta la escolaridad.
    
    \item \textbf{¿Qué variables ayudan a explicar la brecha salarial?} Se analizará cómo factores como edad, región, ocupación y educación influyen en la persistencia de la brecha de ingresos entre géneros.
\end{enumerate}

\subsection{Preguntas Transversales}

\begin{enumerate}
    \item \textbf{¿Los estimadores ponderados difieren significativamente de los no ponderados?} Se validará la importancia de considerar el diseño muestral complejo de CASEN mediante comparación de resultados.
    
    \item \textbf{¿Cómo la metodología de Thompson (2012) se aplica correctamente?} Se documentará la implementación rigorosa de estimadores para diseños estratificados polietápicos, asegurando validez nacional y regional.
\end{enumerate}
