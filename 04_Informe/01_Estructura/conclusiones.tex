% ============================================================================
% 01_Estructura/conclusiones.tex
% ============================================================================

\section{Conclusiones}

Este análisis de la Encuesta CASEN 2022 ha permitido caracterizar dos dimensiones importantes de la desigualdad socioeconómica en Chile.

\subsection{Hallazgos Principales}

\subsubsection{Distribución de la Pobreza}

[Síntesis de hallazgos sobre pobreza por Esteban Román]

\subsubsection{Brecha Salarial de Género}

[Síntesis de hallazgos sobre brecha salarial por Francisca Sepúlveda]

\subsection{Implicaciones}

Los resultados sugieren que... [completar según hallazgos específicos]

\subsection{Limitaciones del Estudio}

\begin{itemize}
    \item CASEN 2022 representa la situación al momento del levantamiento; cambios posteriores no se capturan
    \item Análisis transversal no permite establecer causalidad
    \item Algunas variables pueden tener datos faltantes estratégicamente correlacionados
\end{itemize}

\subsection{Líneas de Investigación Futura}

\begin{itemize}
    \item Análisis dinámico comparando CASEN 2022 con versiones anteriores
    \item Análisis de movilidad socioeconómica
    \item Inclusión de variables no monetarias de bienestar
    \item Análisis de interseccionalidad en desigualdad
\end{itemize}
