% ============================================================================
% 01_Estructura/metodologia.tex
% ============================================================================

\section{Diseño Muestral y Metodología Analítica}

\subsection{Características del Diseño Muestral CASEN 2022}

CASEN 2022 utiliza un \textbf{diseño muestral estratificado polietápico}:

\begin{itemize}
    \item \textbf{Primera etapa:} Estratificación por región (1-16) y zona (urbano/rural)
    \item \textbf{Segunda etapa:} Selección de Unidades Primarias de Muestreo (UPM) - Secciones censales
    \item \textbf{Tercera etapa:} Selección de viviendas y personas dentro de UPM
    \item \textbf{Ponderación:} Variable \texttt{expr} (factor de expansión) ajusta por probabilidades de selección, no-respuesta y calibración
\end{itemize}

Este diseño complejo requiere metodología especializada para estimación insesgada de parámetros poblacionales.

\subsection{Marco Teórico: Thompson (2012)}

Seguimos el marco de Thompson (2012, Capítulos 8-9) para diseños estratificados multietápicos. Para un parámetro poblacional $Y = \sum_i y_i$, el estimador insesgado ponderado es:

\[
\hat{Y} = \sum_i w_i y_i
\]

donde $w_i$ es el peso del elemento $i$ (proporcionalmente igual a \texttt{expr}). Para medias poblacionales:

\[
\hat{\bar{Y}} = \frac{\sum_i y_i w_i}{\sum_i w_i}
\]

La varianza del estimador bajo diseño complejo se calcula considerando estratificación y conglomeración. Implementamos métodos de linearización (Taylor) para construcción de intervalos de confianza.

\subsection{Métodos Analíticos}

\subsubsection*{Análisis Descriptivo Ponderado}

Se calculan:
\begin{itemize}
    \item Promedios y medianas ponderados usando \texttt{expr}
    \item Proporciones ponderadas (e.g., \% de pobres por región)
    \item Comparaciones estratificadas por región, zona (urbano/rural), educación y edad
    \item Deviaciones estándar e intervalos de confianza al 95\% (Wald)
\end{itemize}

\subsubsection*{Tablas Analíticas}

Se generan tablas ponderadas para:
\begin{itemize}
    \item \textbf{Pobreza:} Incidencia de pobreza por región y zona
    \item \textbf{Brecha salarial:} Ingresos comparativos por sexo, educación y edad
    \item \textbf{Análisis estratificado:} Desagregación por subgrupos poblacionales
\end{itemize}

\subsubsection*{Visualización}

Gráficos producen:
\begin{itemize}
    \item Mapeo de distribución geográfica de pobreza
    \item Comparaciones urbano/rural
    \item Evolución de brecha salarial por factores de control
\end{itemize}

\subsection{Control de Calidad}

Se valida:
\begin{itemize}
    \item Coincidencia entre totales ponderados y cifras oficiales CASEN (validación de \texttt{expr})
    \item Ausencia de valores faltantes (missing) en variables clave
    \item Comparación de estimadores ponderados versus no ponderados
    \item Consistencia de tablas cruzadas
\end{itemize}

\subsection{Software y Reproducibilidad}

Todos los análisis son realizados en Python con librerías:
\begin{itemize}
    \item \textbf{pandas:} Manipulación de datos y operaciones ponderadas
    \item \textbf{numpy:} Computación numérica
    \item \textbf{scipy:} Funciones estadísticas y distribuciones
    \item \textbf{matplotlib/seaborn:} Visualización
\end{itemize}

Los scripts y notebooks son completamente reproducibles y disponibles en \texttt{03\_Scripts/} y \texttt{02\_Analisis/}.
