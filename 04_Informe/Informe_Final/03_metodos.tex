% Métodos
\section{Métodos}

\subsection{Fuente de Datos}

\begin{table}[H]
    \centering
    \caption{Características de la fuente de datos}
    \label{tab:fuente}
    \begin{tabular}{ll}
    \toprule
    \tableheadercolor
    Elemento & Descripción \\
    \toprule
    Encuesta & CASEN 2022 (Ministerio de Desarrollo Social y Familia) \\
    Archivo & \texttt{Base\_de\_datos\_Casen\_2022\_STATA} \\
    Universo & Hogares particulares ocupados en Chile \\
    Muestra analítica & 72,056 jefes de hogar (\texttt{pco1 == 1}) \\
    \bottomrule
    \end{tabular}
\end{table}
\subsection{Diseño Muestral}

La CASEN 2022 utiliza un \textbf{diseño muestral probabilístico, estratificado y bietápico}:

\subsubsection{Etapa 1 -- Selección de UPM}

\begin{itemize}
    \item 12,545 Unidades Primarias de Muestreo (conglomerados geográficos)
    \item Selección mediante Probabilidad Proporcional al Tamaño (PPT) sistemática
    \item Estratificación en 764 estratos (Comuna $\times$ Área $\times$ NSE)
\end{itemize}

\subsubsection{Etapa 2 -- Selección de Viviendas}

\begin{itemize}
    \item Muestreo Aleatorio Simple (MAS) dentro de cada UPM
    \item Tamaño total: 106,856 viviendas
\end{itemize}

\subsubsection{Variables del Diseño Muestral}

\begin{table}[H]
    \centering
    \caption{Variables del diseño muestral en R}
    \label{tab:diseno}
    \begin{tabular}{lll}
    \toprule
    \tableheadercolor
    Variable & Código & Descripción \\
    \toprule
    Factor de expansión & \texttt{expr} & Peso regional (persona) \\
    Estrato & \texttt{varstrat} & 764 estratos geográficos \\
    Conglomerado & \texttt{varunit} & 12,545 UPM \\
    \bottomrule
    \end{tabular}
\end{table}

\textbf{Nota metodológica:} Este análisis utiliza un diseño simplificado (\texttt{ids = ~1}) con pesos \texttt{expr}. Esto puede subestimar los errores estándar al ignorar la correlación intra-conglomerado. Para estimaciones de producción se recomienda usar el diseño complejo completo.

\subsection{Variables de Análisis}

\subsubsection{Variables Dependientes}

\begin{table}[H]
    \centering
    \caption{Variables dependientes}
    \label{tab:vdep}
    \begin{tabular}{lll}
    \toprule
    \tableheadercolor
    Variable & Tipo & Construcción \\
    \toprule
    \texttt{es\_pobre} & Dicotómica & 1 si pobreza $\in \{1, 2\}$, 0 si pobreza = 3 \\
    \texttt{es\_pobre\_extremo} & Dicotómica & 1 si pobreza = 1, 0 en otro caso \\
    \texttt{ytrabajocorh} & Continua & Ingreso del trabajo del hogar corregido \\
    \bottomrule
    \end{tabular}
\end{table}

\subsubsection{Variables Independientes}

\begin{table}[H]
    \centering
    \caption{Variables independientes}
    \label{tab:vind}
    \begin{tabular}{lll}
    \toprule
    \tableheadercolor
    Variable & Código CASEN & Valores \\
    \toprule
    Zona & \texttt{area} & 1 = Urbano, 2 = Rural \\
    Sexo & \texttt{sexo} & 1 = Hombre, 2 = Mujer \\
    Escolaridad & \texttt{esc} & Años de educación formal (0--29) \\
    Edad & \texttt{edad} & Años cumplidos \\
    Tamaño hogar & \texttt{tot\_per\_h} & Número de personas \\
    Ocupación & \texttt{oficio4\_08} & Clasificación CIUO-08 \\
    \bottomrule
    \end{tabular}
\end{table}

\subsection{Estrategia Analítica}

\subsubsection{Eje 1 -- Pobreza}

\begin{enumerate}
    \item Estimación de proporciones ponderadas por región y zona
    \item Tests de independencia ($\chi^2$ de diseño complejo)
    \item Modelo logístico (\texttt{quasibinomial}) con odds ratios
    \item Análisis de mediación: Zona $\to$ Educación $\to$ Pobreza (Test de Sobel)
\end{enumerate}

\subsubsection{Eje 2 -- Brecha Salarial}

\begin{enumerate}
    \item Análisis bivariado (brecha bruta sin controles)
    \item Modelo ajustado (controles socioeconómicos: educación, edad)
    \item Modelo completo (controles laborales: ocupación, tamaño hogar)
    \item Modelo log-lineal para interpretación porcentual
\end{enumerate}

\subsection{Software}

Se utilizaron las siguientes librerías en R:

\begin{itemize}
    \item \texttt{tidyverse}: Manipulación de datos
    \item \texttt{survey}: Diseño muestral complejo
    \item \texttt{srvyr}: Interfaz tidyverse para survey
    \item \texttt{haven}: Lectura de archivos .dta
    \item \texttt{scales}: Formateo de números
\end{itemize}
