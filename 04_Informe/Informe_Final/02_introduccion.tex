% Introducción
\section{Introducción}

\subsection{Contexto}

La \textbf{Encuesta de Caracterización Socioeconómica Nacional (CASEN)} es el principal instrumento de medición de pobreza y desigualdad en Chile, administrada por el Ministerio de Desarrollo Social y Familia desde 1985. La versión 2022 permite caracterizar la situación socioeconómica de los hogares chilenos con representatividad nacional, regional y por zona (urbana/rural).

Este proyecto aborda dos ejes de investigación con relevancia para la política pública:

\subsection{Hipótesis}

\subsubsection{Eje A: Distribución Geográfica de la Pobreza}

\begin{description}
    \item[H2a:] La tasa de pobreza en zona rural es significativamente mayor que en zona urbana.
    \item[H2b:] La educación reduce la probabilidad de pobreza, con efecto más pronunciado en zonas urbanas.
\end{description}

\subsubsection{Eje B: Brecha Salarial de Género}

\begin{description}
    \item[H1:] Existe una brecha salarial significativa entre hombres y mujeres (hombre $>$ mujer), incluso controlando por educación, edad, ocupación y composición del hogar.
\end{description}

\subsection{Justificación}

La heterogeneidad territorial de la pobreza en Chile ha sido documentada extensamente (MDSF, 2023), pero los mecanismos que la explican---particularmente el rol mediador de la educación---requieren análisis más profundos. Por otro lado, la brecha salarial de género persiste a pesar de la mayor participación femenina en educación superior, lo que sugiere barreras estructurales en el mercado laboral (INE, 2022).

\subsection{Estructura del Informe}

Este informe está organizado de la siguiente manera:

\begin{itemize}
    \item \textbf{Métodos:} Descripción del diseño muestral, población de estudio, variables y procedimientos analíticos.
    \item \textbf{Resultados:} Presentación de tablas, gráficos y estimaciones ajustadas al diseño muestral.
    \item \textbf{Discusión:} Interpretación de resultados en contexto de antecedentes, fortalezas y limitaciones.
    \item \textbf{Conclusiones:} Síntesis de hallazgos principales y recomendaciones.
\end{itemize}
