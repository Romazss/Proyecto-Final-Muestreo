% Resultados
\section{Resultados}

\subsection{Eje 1: Distribución de la Pobreza}

\subsubsection{Incidencia de pobreza por región}

\begin{table}[H]
    \centering
    \caption{Incidencia de pobreza por región (CASEN 2022)}
    \label{tab:pobreza_region}
    \begin{tabular}{lp{4cm}p{4.5cm}}
    \toprule
    \tableheadercolor
    \textbf{Indicador} & \textbf{Valor} & \textbf{Interpretación} \\
    \midrule
    Tasa pobreza nacional & $\sim$6.5\% & 4,694 jefes en pobreza \\
    Regiones más pobres & Maule, Arica y Parinacota, Magallanes & $\sim$9--10\% (casi el doble) \\
    Regiones menos pobres & La Araucanía, Biobío, Los Ríos & $\sim$2.5--3.5\% (menos de la mitad) \\
    \bottomrule
    \end{tabular}
    
    \vspace{2pt}
    {\small\textit{Nota: Estimaciones ponderadas con factores de expansión regional.}}
\end{table}

Existe una marcada heterogeneidad territorial. Contraintuitivamente, las regiones del Maule, Arica y Parinacota, y zonas extremas como Magallanes concentran las mayores tasas de pobreza ($\sim$9--10\%), mientras que La Araucanía ($\sim$2.5\%), Biobío ($\sim$3\%) y Los Ríos ($\sim$3.5\%) presentan los menores niveles entre jefes de hogar. La Figura \ref{fig:pobreza_region} ilustra esta distribución.

\begin{figure}[H]
    \centering
    \includegraphics[width=0.52\textwidth]{Imagenes/g1_pobreza_region.png}
    \caption{Incidencia de pobreza por región (CASEN 2022)}
    \label{fig:pobreza_region}
\end{figure}

\subsubsection{Pobreza por zona (urbano/rural)}

\begin{table}[H]
    \centering
    \caption{Pobreza por zona urbana y rural}
    \label{tab:pobreza_zona}
    \begin{tabular}{lccc}
    \toprule
    \tableheadercolor
    \textbf{Zona} & \textbf{Proporción} & \textbf{IC 95\%} & \textbf{Diferencia relativa} \\
    \midrule
    Urbano & 5.3\% & [5.0\%, 5.6\%] & Referencia \\
    Rural & 8.4\% & [7.7\%, 9.1\%] & +58\% \\
    \bottomrule
    \end{tabular}
    
    \vspace{2pt}
    {\small\textit{Nota: Diferencia significativa (p $<$ 0.001).}}
\end{table}

La pobreza rural es significativamente mayor que la urbana, con una diferencia relativa del 58\%. La Figura \ref{fig:pobreza_urbano_rural} visualiza esta brecha.

\begin{figure}[H]
    \centering
    \includegraphics[width=0.52\textwidth]{Imagenes/g4_pobreza_urbano_rural.png}
    \caption{Incidencia de pobreza: zona urbana vs rural}
    \label{fig:pobreza_urbano_rural}
\end{figure}

\subsubsection{Modelo logístico de pobreza}

\begin{table}[H]
    \centering
    \caption{Estimaciones del modelo logístico de pobreza}
    \label{tab:modelo_logistico}
    \begin{tabular}{lccp{4cm}}
    \toprule
    \tableheadercolor
    \textbf{Variable} & \textbf{OR} & \textbf{IC 95\%} & \textbf{Interpretación} \\
    \midrule
    Zona rural (ref: urbano) & 1.34 & [1.23, 1.46] & +34\% odds de pobreza \\
    Escolaridad (por año) & 0.93 & [0.91, 0.95] & $-$7\% odds por año \\
    Tamaño hogar (por persona) & 1.21 & [1.18, 1.24] & +21\% odds por persona \\
    \bottomrule
    \end{tabular}
    
    \vspace{2pt}
    {\small\textit{Nota: OR = Odds Ratio; IC = Intervalo de Confianza.}}
\end{table}

La educación es el factor protector más potente. Cada año adicional de escolaridad reduce el odds de pobreza en \textbf{7\%}, controlando por zona y tamaño del hogar (Figura \ref{fig:forest_plot}).

\begin{figure}[H]
    \centering
    \includegraphics[width=0.55\textwidth]{Imagenes/g9_forest_plot_pobreza.png}
    \caption{Forest plot de odds ratios del modelo logístico}
    \label{fig:forest_plot}
\end{figure}

\subsubsection{Análisis de mediación}

\begin{table}[H]
    \centering
    \caption{Descomposición de efectos: Zona $\to$ Educación $\to$ Pobreza}
    \label{tab:mediacion}
    \begin{tabular}{lcp{4.5cm}}
    \toprule
    \tableheadercolor
    \textbf{Efecto} & \textbf{Valor} & \textbf{Interpretación} \\
    \midrule
    Efecto total (c) & 0.49 & Log-odds pobreza rural vs urbano \\
    Efecto directo (c') & 0.27 & Efecto neto controlando educación \\
    Efecto indirecto (a$\times$b) & 0.22 & Vía menor educación rural \\
    \midrule
    \textbf{Proporción mediada} & \textbf{45\%} & Educación explica casi la mitad \\
    Test de Sobel & Z=17.5, p$<$0.001 & Mediación significativa \\
    \bottomrule
    \end{tabular}
\end{table}

La menor escolaridad en zonas rurales media el 45\% del efecto de la ruralidad sobre la pobreza (Figura \ref{fig:mediacion}). Políticas de acceso educativo rural podrían reducir sustancialmente esta brecha.

\begin{figure}[H]
    \centering
    \includegraphics[width=0.52\textwidth]{Imagenes/g10_diagrama_mediacion.png}
    \caption{Diagrama de mediación: Zona $\to$ Educación $\to$ Pobreza}
    \label{fig:mediacion}
\end{figure}

%-------------------------------------------------------------------------------
\subsection{Eje 2: Brecha Salarial de Género}
%-------------------------------------------------------------------------------

\subsubsection{Brecha bruta (sin controles)}

\begin{table}[H]
    \centering
    \caption{Ingresos laborales por sexo (brecha bruta)}
    \label{tab:ingreso_bruto}
    \begin{tabular}{lccc}
    \toprule
    \tableheadercolor
    \textbf{Indicador} & \textbf{Hombre} & \textbf{Mujer} & \textbf{Brecha} \\
    \midrule
    Ingreso medio & \$1,402,711 & \$1,112,358 & \textbf{20.7\%} \\
    Mediana & \$920,000 & \$780,000 & 15.2\% \\
    \bottomrule
    \end{tabular}
    
    \vspace{2pt}
    {\small\textit{Nota: Test t de diseño complejo, p $<$ 0.001.}}
\end{table}

Las jefas de hogar perciben en promedio un 20.7\% menos de ingresos laborales. La brecha en medianas (15.2\%) es menor, sugiriendo que la cola superior de la distribución masculina amplifica la diferencia en medias (Figura \ref{fig:distribucion_ingreso}).

\begin{figure}[H]
    \centering
    \includegraphics[width=0.55\textwidth]{Imagenes/g5_distribucion_ingreso_sexo.png}
    \caption{Distribución de ingresos laborales por sexo}
    \label{fig:distribucion_ingreso}
\end{figure}

\subsubsection{Brecha por nivel educativo}

\begin{table}[H]
    \centering
    \caption{Brecha salarial por nivel educativo}
    \label{tab:brecha_educacion}
    \begin{tabular}{lcp{4.5cm}}
    \toprule
    \tableheadercolor
    \textbf{Nivel educativo} & \textbf{Brecha} & \textbf{Interpretación} \\
    \midrule
    Básica o menos & \textbf{8.2\%} & Menor brecha relativa \\
    Media & 16.9\% & Cerca del promedio nacional \\
    Técnico Superior & \textbf{21.1\%} & Mayor brecha absoluta \\
    Profesional & 14.1\% & Menor que técnico superior \\
    Postgrado & \textbf{20.8\%} & Segunda mayor brecha \\
    \bottomrule
    \end{tabular}
\end{table}

\textbf{Hallazgo clave:} La brecha salarial presenta un patrón \textbf{no lineal} con la educación (Figura \ref{fig:brecha_educacion_combined}). Es mínima en educación básica (8.2\%), aumenta drásticamente en media y técnica superior (16.9\%--21.1\%), disminuye en profesionales (14.1\%), y vuelve a subir en postgrado (20.8\%). Este patrón sugiere que la brecha no se explica únicamente por capital humano.

% Figuras lado a lado: Ingreso por educación y Brecha por educación
\begin{figure}[H]
    \centering
    \begin{subfigure}[b]{0.48\textwidth}
        \centering
        \includegraphics[width=\textwidth]{Imagenes/g2_ingreso_educacion_sexo.png}
        \caption{Ingreso por nivel educativo y sexo}
        \label{fig:ingreso_educacion}
    \end{subfigure}
    \hfill
    \begin{subfigure}[b]{0.48\textwidth}
        \centering
        \includegraphics[width=\textwidth]{Imagenes/g3_brecha_educacion.png}
        \caption{Brecha salarial porcentual}
        \label{fig:brecha_educacion_graf}
    \end{subfigure}
    \caption{Análisis de brecha salarial por nivel educativo}
    \label{fig:brecha_educacion_combined}
\end{figure}

\subsubsection{Modelos de regresión}

\begin{table}[H]
    \centering
    \caption{Modelos de brecha salarial ajustados}
    \label{tab:modelos_brecha}
    \begin{tabular}{lcc}
    \toprule
    \tableheadercolor
    \textbf{Modelo} & \textbf{Coef. Sexo (Mujer)} & \textbf{Interpretación} \\
    \midrule
    Log-lineal (educación, edad) & $-$0.201 (E.E.=0.015) & $-$18.2\% ingreso \\
    Completo (+ ocupación, hogar) & $-$\$198,000 (E.E.=\$28,000) & Brecha absoluta \\
    \bottomrule
    \end{tabular}
    
    \vspace{2pt}
    {\small\textit{Nota: Ambos modelos con p $<$ 0.001. Se rechaza H$_0$.}}
\end{table}

Controlando por educación, edad, ocupación y tamaño del hogar, las jefas de hogar perciben un \textbf{18.2\%} menos de ingreso laboral. Se \textbf{rechaza H$_0$}: existe brecha salarial significativa contra las mujeres incluso con controles completos.

\subsubsection{Brecha por grupo de edad}

\begin{table}[H]
    \centering
    \caption{Brecha salarial por grupo de edad}
    \label{tab:brecha_edad}
    \begin{tabular}{lcp{5cm}}
    \toprule
    \tableheadercolor
    \textbf{Grupo de edad} & \textbf{Brecha} & \textbf{Interpretación} \\
    \midrule
    $\leq$30 años & \textbf{13.1\%} & Menor brecha (entrada al mercado) \\
    31--40 años & 20.8\% & Cerca del promedio ($\sim$20\%) \\
    41--50 años & \textbf{24.5\%} & Alta brecha (pico de carrera) \\
    51--60 años & 17.8\% & Reducción respecto al anterior \\
    60+ años & \textbf{24.8\%} & Mayor brecha (pre-jubilación) \\
    \bottomrule
    \end{tabular}
\end{table}

La brecha salarial presenta un patrón de \textbf{U invertida con rebote} (Figura \ref{fig:brecha_edad}). Es menor al inicio de la carrera laboral (13.1\%), alcanza su máximo en 41--50 años (24.5\%), disminuye en 51--60 (17.8\%), pero vuelve a aumentar en mayores de 60 (24.8\%). Esto sugiere efectos de cohorte y posible discriminación en etapas avanzadas.

\begin{figure}[H]
    \centering
    \includegraphics[width=0.52\textwidth]{Imagenes/g8_brecha_edad.png}
    \caption{Brecha salarial de género por grupo de edad}
    \label{fig:brecha_edad}
\end{figure}

%-------------------------------------------------------------------------------
\subsection{Validación: Sesgo por No Ponderar}
%-------------------------------------------------------------------------------

\begin{table}[H]
    \centering
    \caption{Sesgo por ignorar factores de expansión}
    \label{tab:sesgo_ponderacion}
    \begin{tabular}{lc}
    \toprule
    \tableheadercolor
    \textbf{Estimador} & \textbf{Valor} \\
    \midrule
    Media NO ponderada & \$1,321,458 \\
    Media ponderada & \$1,566,277 \\
    \textbf{Diferencia relativa} & \textbf{+18.5\%} \\
    \bottomrule
    \end{tabular}
\end{table}

Ignorar los factores de expansión subestima el ingreso promedio en $\sim$18.5\%. La muestra CASEN sobrerrepresenta zonas de mayor densidad poblacional, y los pesos corrigen esta distorsión.
