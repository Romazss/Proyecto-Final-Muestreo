% Resultados
\section{Resultados}

\subsection{Eje 1: Distribución de la Pobreza}

\subsubsection{Incidencia de pobreza por región}
\begin{table}[H]
    \centering
    \setlength{\abovecaptionskip}{3pt}
    \setlength{\belowcaptionskip}{3pt}
    \caption{Incidencia de pobreza por región (CASEN 2022)}
    \label{tab:pobreza_region}
    \begin{tabular}{lcc}
    \toprule
    \tableheadercolor
    Indicador & Valor & Interpretación \\
    \toprule
    Tasa pobreza nacional & 6.5\% & 4,694 jefes en pobreza \\
    Región más pobre & La Araucanía/Ñuble & 9.97\% (1.53 veces del promedio) \\
    Región menos pobre & Magallanes & 2.73\% (0.42 veces del promedio) \\
    \bottomrule
    \end{tabular}
    \footnote{Estimaciones ponderadas con factores de expansión regional}
\end{table}
\vspace{-0.3cm}
\begin{figure}[H]
    \centering
    \setlength{\abovecaptionskip}{3pt}
    \setlength{\belowcaptionskip}{3pt}
    \includegraphics[width=0.5\textwidth]{Imagenes/g1_pobreza_region.png}
    \caption{Incidencia de pobreza por región (CASEN 2022)}
    \label{fig:pobreza_region}
\end{figure}
Existe una marcada heterogeneidad territorial. Las regiones del sur y norte presentan las mayores tasas (La Araucanía y Ñuble 9.97\%, Tarapacá 9.36\%, Atacama 7.55\%), mientras que Magallanes presenta la menor incidencia (2.73\%). La Región Metropolitana también muestra tasas bajas (3.73\%), reflejando concentración de pobreza en zonas periféricas.

\subsubsection{Pobreza por zona (urbano/rural)}
\begin{table}[H]
    \centering
    \setlength{\abovecaptionskip}{3pt}
    \setlength{\belowcaptionskip}{3pt}
    \caption{Pobreza por zona urbana y rural}
    \label{tab:pobreza_zona}
    \begin{tabular}{lccc}
    \toprule
    \tableheadercolor
    Zona & Proporción & IC 95\% & Diferencia relativa \\
    \toprule
    Urbano & 5.3\% & [5.0\%, 5.6\%] & Referencia \\
    Rural & 8.4\% & [7.7\%, 9.1\%] & +58\% \\
    \bottomrule
    \end{tabular}
    \footnote{Diferencia significativa (p $<$ 0.001)}
\end{table}

\begin{figure}[H]
    \centering
    \includegraphics[width=0.8\textwidth]{Imagenes/g4_pobreza_urbano_rural.png}
    \caption{Incidencia de pobreza: zona urbana vs rural}
    \label{fig:pobreza_urbano_rural}
\end{figure}

\subsubsection{Modelo logístico de pobreza}

\begin{table}[H]
    \centering
    \caption{Estimaciones del modelo logístico de pobreza}
    \label{tab:modelo_logistico}
    \begin{tabular}{lccp{4cm}}
    \toprule
    \tableheadercolor
    Variable & OR & IC 95\% & Interpretación \\
    \toprule
    Zona rural (ref: urbano) & 1.34 & [1.23, 1.46] & +34\% odds de pobreza \\
    Escolaridad (por año) & 0.94 & [0.93, 0.95] & --6\% odds por año adicional \\
    Tamaño hogar (por persona) & 1.22 & [1.19, 1.25] & +22\% odds por persona adicional \\
    \bottomrule
    \end{tabular}
    \footnote{OR: Odds Ratio; IC: Intervalo de Confianza}
\end{table}

\begin{figure}[H]
    \centering
    \includegraphics[width=0.8\textwidth]{Imagenes/g9_forest_plot_pobreza.png}
    \caption{Forest plot de odds ratios del modelo logístico}
    \label{fig:forest_plot}
\end{figure}

La educación es el factor protector más potente. Cada año adicional de escolaridad reduce el odds de pobreza en 6\%, controlando por zona y tamaño del hogar.

\subsubsection{Análisis de mediación}

\begin{table}[H]
    \centering
    \caption{Descomposición de efectos: Zona $\to$ Educación $\to$ Pobreza}
    \label{tab:mediacion}
    \begin{tabular}{lcp{5cm}}
    \toprule
    \tableheadercolor
    Efecto & Valor & Interpretación \\
    \toprule
    Efecto total (c) & 0.49 & Log-odds de pobreza rural vs urbano \\
    Efecto directo (c') & 0.27 & Efecto neto controlando educación \\
    Efecto indirecto (a$\times$b) & 0.22 & Vía menor educación en zonas rurales \\
    \midrule
    \textbf{Proporción mediada} & \textbf{45\%} & La educación explica casi la mitad \\
    Test de Sobel & Z=17.5, p$<$0.001 & Mediación significativa \\
    \bottomrule
    \end{tabular}
\end{table}

\begin{figure}[H]
    \centering
    \includegraphics[width=0.8\textwidth]{Imagenes/g10_diagrama_mediacion.png}
    \caption{Diagrama de mediación: Zona $\to$ Educación $\to$ Pobreza}
    \label{fig:mediacion}
\end{figure}

La menor escolaridad en zonas rurales media el 45\% del efecto de la ruralidad sobre la pobreza. Políticas de acceso educativo rural podrían reducir sustancialmente esta brecha.

\subsection{Eje 2: Brecha Salarial de Género}

\subsubsection{Brecha bruta (sin controles)}

\begin{table}[H]
    \centering
    \caption{Ingresos laborales por sexo (brecha bruta)}
    \label{tab:ingreso_bruto}
    \begin{tabular}{lccc}
    \toprule
    \tableheadercolor
    Indicador & Hombre & Mujer & Brecha \\
    \toprule
    Ingreso medio & \$1,402,711 & \$1,112,358 & \textbf{20.7\%} \\
    Mediana & \$920,000 & \$780,000 & 15.2\% \\
    \bottomrule
    \end{tabular}
    \footnote{Test t de diseño complejo: p $<$ 0.001}
\end{table}

\begin{figure}[H]
    \centering
    \includegraphics[width=0.8\textwidth]{Imagenes/g5_distribucion_ingreso_sexo.png}
    \caption{Distribución de ingresos laborales por sexo}
    \label{fig:distribucion_ingreso}
\end{figure}

Las jefas de hogar perciben en promedio un 20.7\% menos de ingresos laborales. La brecha en medianas (15.2\%) es menor, sugiriendo que la cola superior de la distribución masculina amplifica la diferencia en medias.

\subsubsection{Brecha por nivel educativo}

\begin{table}[H]
    \centering
    \caption{Brecha salarial por nivel educativo}
    \label{tab:brecha_educacion}
    \begin{tabular}{lccc}
    \toprule
    \tableheadercolor
    Nivel educativo & Hombre & Mujer & Brecha \\
    \toprule
    Básica (esc $\leq$ 8) & \$680,000 & \$560,000 & 17\% \\
    Media (esc 9--12) & \$920,000 & \$750,000 & 18\% \\
    Superior (esc 13--16) & \$1,450,000 & \$1,130,000 & 22\% \\
    Postgrado (esc $>$ 16) & \$2,800,000 & \$2,100,000 & \textbf{25\%} \\
    \bottomrule
    \end{tabular}
\end{table}

\begin{figure}[H]
    \centering
    \includegraphics[width=0.8\textwidth]{Imagenes/g2_ingreso_educacion_sexo.png}
    \caption{Ingreso laboral por nivel educativo y sexo}
    \label{fig:ingreso_educacion}
\end{figure}

\textbf{Hallazgo clave:} La brecha salarial \textbf{aumenta} con el nivel educativo. Las mujeres con postgrado enfrentan la mayor brecha relativa ($\sim$25\%), contradiciendo la hipótesis de que más educación iguala ingresos.

\subsubsection{Modelo log-lineal ajustado}

\textbf{Especificación:} $\log(\texttt{ytrabajocorh}) \sim \text{factor(sexo)} + \text{esc} + \text{edad} + \text{I(edad}^2\text{)}$

\begin{table}[H]
    \centering
    \caption{Coeficientes del modelo log-lineal}
    \label{tab:modelo_log}
    \begin{tabular}{lccc}
    \toprule
    \tableheadercolor
    Parámetro & Coeficiente & E.E. & Interpretación \\
    \toprule
    Sexo (Mujer) & -0.201 & 0.015 & -18.2\% de ingreso \\
    \bottomrule
    \end{tabular}
    \footnote{Controlando por educación, edad y edad$^2$}
\end{table}
\begin{figure}[H]
    \centering
    \includegraphics[width=0.35\textwidth]{Imagenes/g3_brecha_educacion.png}
    \caption{Brecha salarial porcentual por nivel educativo}
    \label{fig:brecha_educacion_graf}
\end{figure}

Controlando por educación, edad y edad$^2$, las jefas de hogar perciben un \textbf{18.2\%} menos de ingreso laboral que los jefes hombres.

\subsubsection{Modelo completo (con ocupación)}

\begin{table}[H]
    \centering
    \caption{Modelo con ocupación como variable control}
    \label{tab:modelo_completo}
    \begin{tabular}{lc}
    \toprule
    \tableheadercolor
    Estadístico & Valor \\
    \toprule
    $\beta_{\text{sexo}}$ & --\$198,000 \\
    Error estándar & \$28,000 \\
    p-valor & $<$ 0.001 \\
    \bottomrule
    \end{tabular}
    \footnote{Controles: ocupación (CIUO-08), educación, edad, tamaño del hogar}
\end{table}

Se \textbf{rechaza H$_0$}. Existe brecha salarial significativa contra las mujeres, incluso controlando por ocupación, educación, edad y tamaño del hogar.

\subsubsection{Brecha por grupo de edad}

\begin{figure}[H]
    \centering
    \includegraphics[width=0.8\textwidth]{Imagenes/g8_brecha_edad.png}
    \caption{Brecha salarial de género por grupo de edad}
    \label{fig:brecha_edad}
\end{figure}

\subsection{Validación: Sesgo por No Ponderar}

\begin{table}[H]
    \centering
    \caption{Sesgo por ignorar factores de expansión}
    \label{tab:sesgo_ponderacion}
    \begin{tabular}{lc}
    \toprule
    \tableheadercolor
    Estimador & Valor \\
    \toprule
    Media NO ponderada & \$1,321,458 \\
    Media ponderada & \$1,566,277 \\
    Diferencia relativa & +18.5\% \\
    \bottomrule
    \end{tabular}
\end{table}

Ignorar los factores de expansión subestima el ingreso promedio en $\sim$18.5\%. La muestra CASEN sobrerrepresenta zonas de mayor densidad poblacional, y los pesos corrigen esta distorsión.
