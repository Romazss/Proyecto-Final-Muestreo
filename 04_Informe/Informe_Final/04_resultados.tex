% Resultados
\section{Resultados}

\subsection{Características de la Muestra}

\subsubsection{Descriptiva General}

% Tabla 1: Características demográficas de la muestra
% Incluir frecuencias, porcentajes y errores estándar

\begin{table}[H]
    \centering
    \caption{Características demográficas de la muestra}
    \label{tab:caracteristicas}
    \begin{tabular}{lcc}
    \toprule
    Variable & Frecuencia & Porcentaje (\%) \\
    \midrule
    % Completar con datos
    \bottomrule
    \end{tabular}
    \footnote{Estimaciones ponderadas considerando el diseño muestral}
\end{table}

\subsection{Resultados Principales}

\subsubsection{Estimaciones Ajustadas al Diseño Muestral}

% Tabla 2: Estimaciones principales con intervalos de confianza
% Incluir errores estándar y factores de diseño

\begin{table}[H]
    \centering
    \caption{Estimaciones principales ajustadas al diseño muestral}
    \label{tab:estimaciones}
    \begin{tabular}{lcccc}
    \toprule
    Estimador & Estimación & E.E. & IC 95\% \\
    \midrule
    % Completar con datos
    \bottomrule
    \end{tabular}
    \footnote{IC: Intervalo de Confianza; E.E.: Error Estándar}
\end{table}

\subsubsection{Análisis por Subgrupos}

% Presentar resultados estratificados por variables clave

\begin{table}[H]
    \centering
    \caption{Estimaciones por subgrupos}
    \label{tab:subgrupos}
    \begin{tabular}{lcccc}
    \toprule
    Subgrupo & Estimación & E.E. & IC 95\% \\
    \midrule
    % Completar con datos
    \bottomrule
    \end{tabular}
\end{table}

\subsection{Visualización de Resultados}

\subsubsection{Gráficos}

% Figura 1: Gráfico comparativo

\begin{figure}[H]
    \centering
    \includegraphics[width=0.8\textwidth]{figures/grafico1.pdf}
    \caption{Descripción del gráfico 1}
    \label{fig:grafico1}
\end{figure}

% Figura 2: Gráfico de tendencias

\begin{figure}[H]
    \centering
    \includegraphics[width=0.8\textwidth]{figures/grafico2.pdf}
    \caption{Descripción del gráfico 2}
    \label{fig:grafico2}
\end{figure}

\subsection{Evaluación de Precisión}

% Tabla con factores de diseño y coeficientes de variación

\begin{table}[H]
    \centering
    \caption{Evaluación de precisión - Factores de diseño}
    \label{tab:precision}
    \begin{tabular}{lccc}
    \toprule
    Variable & DEFF & CV (\%) & Tamaño efectivo \\
    \midrule
    % Completar con datos
    % DEFF: Design Effect (Efecto del diseño)
    % CV: Coeficiente de Variación
    \bottomrule
    \end{tabular}
    \footnote{DEFF: Efecto del diseño; CV: Coeficiente de Variación}
\end{table}

\subsection{Modelos Analíticos (si aplica)}

% Resultados de regresión u otros modelos considerando el diseño muestral

\begin{table}[H]
    \centering
    \caption{Resultados de modelo de regresión}
    \label{tab:modelo}
    \begin{tabular}{lccc}
    \toprule
    Variable & Coeficiente & E.E. & Valor-p \\
    \midrule
    % Completar con datos
    \bottomrule
    \end{tabular}
\end{table}
