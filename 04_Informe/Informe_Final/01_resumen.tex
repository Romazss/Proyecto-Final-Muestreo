% Resumen del proyecto
\section*{Resumen}
\addcontentsline{toc}{section}{Resumen}

\begin{hallazgobox}

Este estudio analiza dos fenómenos de desigualdad socioeconómica en Chile utilizando datos de la \textbf{Encuesta CASEN 2022} (n = 72,056 jefes de hogar): (1) la distribución geográfica de la pobreza y (2) la brecha salarial de género. Empleando un \textbf{diseño muestral complejo completo} con estratificación (\texttt{varstrat}), conglomerados (\texttt{varunit}) y factores de expansión (\texttt{expr}), estimamos proporciones, medias y modelos de regresión con errores estándar robustos.

\vspace{0.2cm}
\noindent
\textbf{Principales hallazgos:}

\begin{itemize}
    \item La pobreza rural (8.37\%, IC: [7.79\%, 8.95\%]) es significativamente mayor que la urbana (5.28\%, IC: [5.04\%, 5.52\%]), con una diferencia de 3.09 pp ($\chi^2$ Rao-Scott = 131.79, p $<$ 0.001).
    \item La menor escolaridad en zonas rurales \textbf{media el 44.9\%} del efecto de la ruralidad sobre la pobreza (Test de Sobel, Z = 16.80, p $<$ 0.001).
    \item Existe una brecha salarial de género del \textbf{20.7\%} en términos brutos (\$290,353 de diferencia, t = $-$14.53, p $<$ 0.001), que se reduce a \textbf{18.2\%} controlando por educación, edad y ocupación.
    \item La brecha salarial presenta un patrón \textbf{no lineal} con la educación: mínima en básica (8.2\%), máxima en técnico superior (21.1\%) y postgrado (20.8\%).
    \item Ignorar los factores de expansión subestima el ingreso promedio en un \textbf{18.5\%} (\$1,321,458 vs \$1,566,277).
\end{itemize}

\vspace{0.2cm}
\noindent
\textbf{Palabras clave:} CASEN 2022, pobreza, brecha salarial, diseño muestral complejo, mediación estadística

\end{hallazgobox}
