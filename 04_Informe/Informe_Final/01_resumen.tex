% Resumen del proyecto
\section*{Resumen}
\addcontentsline{toc}{section}{Resumen}

\begin{hallazgobox}

Este estudio analiza dos fenómenos de desigualdad socioeconómica en Chile utilizando datos de la \textbf{Encuesta CASEN 2022}: (1) la distribución geográfica de la pobreza y (2) la brecha salarial de género. Empleando técnicas de muestreo complejo con factores de expansión regional, estimamos proporciones, medias y modelos de regresión ajustados al diseño muestral.

\vspace{0.2cm}
\noindent
\textbf{Principales hallazgos:}

\begin{itemize}
    \item La pobreza rural (8.4\%) es significativamente mayor que la urbana (5.3\%), con una diferencia relativa del 58\%.
    \item La menor escolaridad en zonas rurales \textbf{media el 45\%} del efecto de la ruralidad sobre la pobreza (Test de Sobel, Z=17.5, p$<$0.001).
    \item Existe una brecha salarial de género del \textbf{20.7\%} en términos brutos, que se reduce a \textbf{18.2\%} controlando por educación, edad y ocupación.
    \item Paradójicamente, la brecha salarial \textbf{aumenta} con el nivel educativo, alcanzando $\sim$25\% en personas con postgrado.
\end{itemize}

\vspace{0.2cm}
\noindent
\textbf{Palabras clave:} CASEN 2022, pobreza, brecha salarial, diseño muestral complejo, mediación estadística

\end{hallazgobox}
