% Conclusiones
\section{Conclusiones}

\subsection{Principales Hallazgos}

\begin{hallazgobox}

\noindent
\textbf{Hallazgo 1:} La pobreza rural es 58\% mayor que la urbana (8.4\% vs 5.3\%, p $<$ 0.001), con heterogeneidad regional significativa: La Araucanía y Ñuble ($\sim$10\%) vs Magallanes ($\sim$2.7\%) y Aysén ($\sim$3.4\%).

\vspace{0.4cm}

\noindent
\textbf{Hallazgo 2:} La educación media el 45\% del efecto de la zona sobre la pobreza. Cada año adicional de escolaridad reduce el odds de pobreza en \textbf{7\%}.

\vspace{0.4cm}

\noindent
\textbf{Hallazgo 3:} Existe una brecha salarial de género del 18--21\% que persiste controlando por educación, edad, ocupación y tamaño del hogar.

\vspace{0.4cm}

\noindent
\textbf{Hallazgo 4:} La brecha salarial presenta un patrón no lineal: mínima en educación básica (8.2\%), máxima en técnico superior (21.1\%) y postgrado (20.8\%), evidenciando barreras estructurales más allá del capital humano.

\vspace{0.4cm}

\noindent
\textbf{Hallazgo 5:} No ponderar los datos introduce un sesgo del 18.5\% en la estimación de ingresos, destacando la importancia de respetar el diseño muestral.

\end{hallazgobox}

\subsection{Respuesta a Hipótesis}

\subsubsection{Eje A: Distribución Geográfica de la Pobreza}

\begin{description}
    \item[H2a:] \textbf{Confirmada.} La tasa de pobreza rural (8.4\%) es significativamente mayor que urbana (5.3\%), con diferencia relativa de 58\%.
    
    \item[H2b:] \textbf{Confirmada.} La educación actúa como factor protector potente (OR = 0.93 por año, $-$7\% por año adicional), con rol mediador central en explicar la brecha territorial.
\end{description}

\subsubsection{Eje B: Brecha Salarial de Género}

\begin{description}
    \item[H1:] \textbf{Confirmada.} Existe brecha salarial de género significativa (18--21\%) incluso controlando por educación, edad, ocupación y tamaño del hogar, evidenciando discriminación residual.
\end{description}

\subsection{Conclusión General}

\begin{hallazgobox}

Los análisis confirman patrones estructurales de desigualdad territorial y de género en Chile, donde la educación emerge como factor protector contra la pobreza pero insuficiente para cerrar la brecha salarial entre hombres y mujeres. Estos resultados tienen implicaciones inmediatas para política pública: la inversión educativa rural y las políticas de transparencia salarial constituyen puntos de apalancamiento clave para reducir estas desigualdades.

\end{hallazgobox}

\subsection{Recomendaciones}

\subsubsection{Para Política Pública}

\begin{enumerate}
    \item Focalizar inversión educativa en regiones de alta pobreza: La Araucanía ($\sim$10\%), Ñuble ($\sim$10\%) y Tarapacá ($\sim$9.4\%)
    \item Implementar mecanismos de transparencia salarial para reducir discriminación, especialmente en ocupaciones técnicas superiores y de postgrado
    \item Evaluar el impacto de políticas de corresponsabilidad parental en la reducción de la brecha de género
\end{enumerate}

\subsubsection{Para Investigación Futura}

\begin{enumerate}
    \item Incorporar análisis longitudinal con panel CASEN para validar causalidad
    \item Descomponer la brecha salarial utilizando métodos Oaxaca-Blinder por sector
    \item Explorar roles de discriminación versus segregación ocupacional en la brecha
\end{enumerate}
