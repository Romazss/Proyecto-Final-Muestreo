% Discusión
\section{Discusión}

\subsection{Síntesis de Hallazgos}

Los resultados confirman dos patrones de desigualdad estructural en Chile:

\begin{enumerate}
    \item \textbf{Brecha territorial:} La pobreza rural supera significativamente a la urbana, con la educación como mecanismo mediador central. Este hallazgo es consistente con la literatura sobre desarrollo territorial desigual en América Latina.
    
    \item \textbf{Brecha de género:} La diferencia salarial persiste incluso controlando por capital humano y características laborales, evidenciando discriminación residual en el mercado laboral.
\end{enumerate}

\subsection{Comparación con Antecedentes}

\subsubsection{Pobreza Territorial}

Nuestros resultados (5.3\% urbano vs 8.4\% rural) son consistentes con las estimaciones oficiales del MDSF (2023) y confirman la tendencia histórica de mayor incidencia rural documentada desde CASEN 1990.

\subsubsection{Brecha Salarial}

La brecha bruta de 20.7\% es similar a la reportada por el INE (2022) para el ingreso medio mensual. El hallazgo de que la brecha presenta un \textit{patrón no lineal} con la educación---siendo máxima en educación técnica superior (21.1\%) y postgrado (20.8\%), pero menor en profesionales (14.1\%)---es consistente con estudios internacionales sobre el \textit{glass ceiling} (techo de cristal) y la segregación ocupacional que afecta especialmente a ocupaciones técnicas y de alta dirección.

Según el INE:

\begin{quote}
``Se confirma que la segregación horizontal y vertical del mercado del trabajo según el sexo de las personas están presentes en el mercado laboral en Chile: existen ramas altamente feminizadas... además, en cada una de las ramas analizadas las mujeres acceden en menor medida que los hombres a posiciones de mayor jerarquía.''
\end{quote}

\subsection{Limitaciones}

\subsubsection{Limitaciones Muestrales}

\begin{itemize}
    \item \textbf{Diseño muestral simplificado:} El uso de \texttt{ids = ~1} puede subestimar los errores estándar. Para inferencias más precisas se recomienda incorporar estratos y conglomerados.
    
    \item \textbf{Sesgo de selección:} El análisis de brecha salarial excluye a quienes no participan en el mercado laboral, potencialmente subestimando la desigualdad total.
\end{itemize}

\subsubsection{Limitaciones Analíticas}

\begin{itemize}
    \item \textbf{Variable de ingreso:} \texttt{ytrabajocorh} representa el ingreso del hogar, no del individuo. Esto puede introducir confusión en hogares con múltiples perceptores.
    
    \item \textbf{Corte transversal:} Los datos no permiten establecer causalidad ni controlar por factores no observados constantes en el tiempo.
    
    \item \textbf{Manejo de NA:} Aunque se implementó \texttt{case\_when()} para evitar sesgos, la proporción de datos faltantes en algunas variables puede afectar la representatividad.
\end{itemize}

\subsection{Fortalezas del Estudio}

\begin{enumerate}
    \item Representatividad nacional con cobertura regional y urbano-rural
    \item Aplicación rigurosa de técnicas de diseño muestral complejo
    \item Análisis de mediación para identificar mecanismos causales
    \item Triangulación de métodos (descriptivos, modelos ajustados, descomposición)
\end{enumerate}

\subsection{Sugerencias para Futuras Investigaciones}

\begin{enumerate}
    \item Incorporar análisis longitudinal con panel CASEN para establecer causalidad
    \item Explorar heterogeneidad de la brecha por sector económico y región
    \item Implementar métodos de descomposición (Oaxaca-Blinder) para brecha salarial
    \item Evaluar políticas de corresponsabilidad parental y su impacto en la brecha
\end{enumerate}

\subsection{Implicaciones para Política Pública}

\begin{itemize}
    \item \textbf{Reducción de pobreza:} Focalizar inversión educativa en zonas rurales como estrategia de reducción de pobreza, dada su capacidad mediadora (45\% del efecto)
    
    \item \textbf{Equidad salarial:} Revisar mecanismos de transparencia salarial para reducir discriminación, particularmente en ocupaciones de mayor calificación donde la brecha se amplifica
    
    \item \textbf{Desarrollo territorial:} Integrar en políticas públicas el reconocimiento de la heterogeneidad regional, con énfasis en regiones de alta pobreza: Maule ($\sim$10\%), Arica y Parinacota ($\sim$9\%) y Magallanes ($\sim$10\%)
\end{itemize}
